% arara: xelatex: {shell: yes}
%% arara: biber
%% arara: xelatex: {shell: yes}
%% arara: xelatex: {shell: yes}


\documentclass[11pt, a4paper]{article}
\usepackage{libertine}

\usepackage{fontspec}

\usepackage[base]{babel} % hypenation
% see: https://tex.stackexchange.com/questions/400986/hyphenrules-environment-no-longer-works-with-polyglossia
\usepackage{polyglossia}

% \setdefaultlanguage{russian}
\setmainlanguage{english}
\setotherlanguages{russian}

% download "Linux Libertine" OTF-fonts:
% http://www.linuxlibertine.org/index.php?id=91&L=1
% \setmainfont{Linux Libertine O} % or Helvetica, Arial, Cambria
% why do we need \newfontfamily:
% http://tex.stackexchange.com/questions/91507/
% \newfontfamily{\cyrillicfonttt}{Linux Libertine O}
% \newfontfamily{\cyrillicfont}{Linux Libertine O}
% \newfontfamily{\cyrillicfontsf}{Linux Libertine O}
 
\usepackage{etoolbox} % to use ifdef, must be after babel

\newtoggle{excerpt}
\togglefalse{excerpt}  % помечаем, что это не отрывок, а далее в тексте может использовать
 

\usepackage[paper=a4paper,
top=15mm,
bottom=13.5mm,
left=13mm, right=13mm, includefoot]{geometry}

%\usepackage{etex} % расширение классического tex
% extended allocation already in use?
% в частности позволяет подгружать гораздо больше пакетов, чем мы и займёмся далее




\usepackage{makeidx} % для создания предметных указателей
\usepackage{verbatim} % для многострочных комментариев
%\usepackage[pdftex]{graphicx} % для вставки графики
% omit pdftex option if not using pdflatex

\usepackage{comment} % для команды excludecomment


%\usepackage{dsfont} % шрифт для единички с двойной палочкой (для индикатора события)
\usepackage{bbm} % шрифт - двойные буквы


\usepackage[usenames, dvipsnames, svgnames, table, rgb]{xcolor}

\usepackage{colortbl}


% пакет для тестов:
% \usepackage[box, % запрет на перенос вопросов
% nopage, % убираем колонтитулы страницы
% insidebox, % ставим буквы в квадратики
% separateanswersheet, % добавляем бланк ответов
% nowatermark, % отсутствие надписи "Черновик"
% indivanswers,  % показываем верные ответы
% answers,
% lang=RU, % локализация слов "нет верных ответов", "вопрос" и тд
% completemulti % добавлять "нет правильного ответа" во всех вопросах множественного выбора
% ]{automultiplechoice}


\usepackage[colorlinks, hyperindex, unicode, breaklinks]{hyperref} % гиперссылки в pdf

\usepackage{amssymb}
\usepackage{amsmath}
\usepackage{amsthm}
\usepackage{epsfig}
\usepackage{bm}
\usepackage{color}

\usepackage{multicol}
\usepackage{multirow} % Слияние строк в таблице

\usepackage{textcomp}  % Чтобы в формулах можно было русские буквы писать через \text{}

%\usepackage{embedfile} % отказались от внедрения тех внутрь pdf так как всё равно всё на гитхабе :)

\usepackage{physics} % \abs \norm \grad, меняет \sin, \cos...

\usepackage{subfigure} % для создания нескольких рисунков внутри одного

\usepackage{tikz, pgfplots} % язык для рисования графики из latex'a
\usetikzlibrary{trees} % прибамбас в нем для рисовки деревьев
\usetikzlibrary{arrows} % прибамбас в нем для рисовки стрелочек подлиннее
\usetikzlibrary{automata, positioning, calc}
\usepackage{tikz-qtree} % прибамбас в нем для рисовки деревьев

\pgfplotsset{compat=1.18} 
% otherwise we get message 
% running in backwards compatibility mode (unsuitable tick labels; missing features). 





\usepackage{enumitem} % развернутые списки

% свешиваем пунктуацию (т.е. знаки пунктуации могут вылезать за правую границу текста, при этом текст выглядит ровнее)
\usepackage{microtype}

% более красивые таблицы
\usepackage{booktabs}
% заповеди из докупентации:
% 1. Не используйте вертикальные линни
% 2. Не используйте двойные линии
% 3. Единицы измерения - в шапку таблицы
% 4. Не сокращайте .1 вместо 0.1
% 5. Повторяющееся значение повторяйте, а не говорите "то же"

\usepackage[cache=false]{minted} % вставка кода, нужен питон :)
% опция cache=false включена, чтобы избегать необходимости
% чистить кэш при ошибках компиляции
% возможно вообще подумать об устранении этого пакета:
% вставок кода мало, а эта зависимость (pygmentize + python)
% резко затрудняет редактирование новичкам
% может обойтись listings?

\usepackage{epigraph}

\usepackage{titleps} % заголовки страниц




% по поводу заголовков разделов в колонтитулах
% https://tex.stackexchange.com/questions/236705
% поэтому выбрали titleps вместо fancyhdr

\newpagestyle{mypage}{%
  \headrule
  \sethead{\sectiontitle}{}{\subsectiontitle}
  \setfoot{}{}{\thepage}
}
\settitlemarks{section,subsection,subsubsection} % !!!!!!no space after comma!!!!!!
\pagestyle{mypage}





\DeclareMathOperator{\Lin}{\mathrm{Lin}}
\DeclareMathOperator{\Linp}{\Lin^{\perp}}
\DeclareMathOperator*\plim{plim}
\DeclareMathOperator{\card}{card}
\DeclareMathOperator{\sgn}{sign}
\DeclareMathOperator{\sign}{sign}

\DeclareMathOperator*{\argmin}{arg\,min}
\DeclareMathOperator*{\argmax}{arg\,max}
\DeclareMathOperator*{\amn}{arg\,min}
\DeclareMathOperator*{\amx}{arg\,max}
\DeclareMathOperator{\cov}{Cov}
\DeclareMathOperator{\Var}{Var}
\DeclareMathOperator{\Cov}{Cov}
\DeclareMathOperator{\Corr}{Corr}
\DeclareMathOperator{\pCorr}{pCorr}
\DeclareMathOperator{\E}{\mathbb{E}}
\let\P\relax
\DeclareMathOperator{\P}{\mathbb{P}}


\newcommand{\cN}{\mathcal{N}}
\newcommand{\dU}{\mathrm{U}}
\newcommand{\dUnif}{\mathrm{U}}
\newcommand{\dBinom}{\mathrm{Bin}}
\newcommand{\dBin}{\dBinom}
\newcommand{\dExp}{\mathrm{Exp}}
\newcommand{\dPois}{\mathrm{Pois}}
\newcommand{\dBeta}{\mathrm{Beta}}
\newcommand{\dGamma}{\mathrm{Gamma}}

\newcommand{\score}{\mathrm{score}}
\newcommand{\crit}{\mathrm{crit}}


\newcommand \R{\mathbb{R}}
\newcommand \N{\mathbb{N}}
\newcommand \Z{\mathbb{Z}}


\newcommand \RR{\mathbb{R}}
\newcommand \NN{\mathbb{N}}
\newcommand \ZZ{\mathbb{Z}}




\newcommand{\dx}[1]{\,\mathrm{d}#1} % для интеграла: маленький отступ и прямая d
\newcommand{\ind}[1]{\mathbbm{1}_{\{#1\}}} % Индикатор события
%\renewcommand{\to}{\rightarrow}
\newcommand{\eqdef}{\mathrel{\stackrel{\rm def}=}}
\newcommand{\iid}{\mathrel{\stackrel{\rm i.\,i.\,d.}\sim}}
\newcommand{\const}{\mathrm{const}}


% вместо горизонтальной делаем косую черточку в нестрогих неравенствах
\renewcommand{\le}{\leqslant}
\renewcommand{\ge}{\geqslant}
\renewcommand{\leq}{\leqslant}
\renewcommand{\geq}{\geqslant}



\AddEnumerateCounter{\asbuk}{\russian@alph}{щ} % для списков с русскими буквами
% \setlist[enumerate, 2]{label=\asbuk*),ref=\asbuk*} % \asbuk* \alph* \arabic*




% делаем короче интервал в списках
\setlength{\itemsep}{0pt}
\setlength{\parskip}{0pt}
\setlength{\parsep}{0pt}

% \newenvironment{problem}{}{}
% тут перещёлкиваем комментарий, чтобы убрать или показать решения:
% \newenvironment{sol}{}{} % with solutions
% \excludecomment{sol} % without solutions



\unitlength=0.6mm

\title{Time Series and Stochastic Processes exams}
\date{\today}
\author{Angry Teachers, Folklore}


%%%%%%%%%%%%%%%%%% вставки
%%%%%%%%%%%%%%%%%%%%%%%%%%%%%%%%%%%%%%% Списки без уродских отступов
\newenvironment{enumerate*}{
\begin{enumerate}
  \setlength{\itemsep}{0pt}
  \setlength{\parskip}{0pt}
  \setlength{\parsep}{0pt}
}{\end{enumerate}}

\newenvironment{itemize*}{
\begin{itemize}
  \setlength{\itemsep}{0pt}
  \setlength{\parskip}{0pt}
  \setlength{\parsep}{0pt}
}{\end{itemize}}

\abovedisplayskip=0mm
\abovedisplayshortskip=0mm
\belowdisplayskip=0mm
\belowdisplayshortskip=0mm


% https://tex.stackexchange.com/questions/241343
% https://tex.stackexchange.com/questions/168480
\emergencystretch 5em % разрешаем дополнительные пробелы для упаковки параграфа до правой границы


% \setcounter{secnumdepth}{0} % убираем нумерацию секций, подсекций и т.д.


%%%%%%%%%%%
% блок для тестов
%%%%%%%%%%%
% [1][3] 1 = one argument, 3 = value if missing
% эта магия создаёт окружение answerlist
% именно в окружении answerlist записаны варианты ответов в подключаемых exerciseXX
% просто \begin{answerlist} сделает ответы в три столбца
% если ответы длинные, то надо в них руками сделать
% \begin{answerlist}[1] чтобы они шли в один столбец
\newenvironment{answerlist}[1][3]{
\begin{multicols}{#1}
\begin{enumerate}[label=\fbox{\emph{\Alph*}},ref=\emph{\alph*}]
}
{
\end{enumerate}
\end{multicols}
}


\excludecomment{solution} % without solutions

\theoremstyle{definition}

% опция [subsection] для сброса счётчика вопросов после каждой subsection
\newtheorem{question}{Вопрос}[subsection]

% чтобы номер вопроса был без номера секции:
\renewcommand{\thequestion}{\arabic{question}}
% конец блока для тестов
%%%%%%%%%%%%









\begin{document}
\maketitle

%\clearpage
%\thispagestyle{empty}
\tableofcontents{}


\parindent=0 pt % no indent

\clearpage
\section*{Description}

See updates at \url{https://github.com/bdemeshev/tssp_exams}.

Many more problems can be found at  \url{https://github.com/bdemeshev/stochastic_pro}.

Click on red hyperlinks inside pdf, you can get to the answers and back!


Any comments? Bugs?
\url{https://github.com/bdemeshev/tssp_hse_exams/issues/}.

The order of topics has changed substantionally after the first course iteration in 2020-21.
The interested reader may find relevant exercises by looking through all 2020-21 exams. 


\subsection*{Greatings to the contributors}

Here we describe only the style guidelines and typical erros. 
For more information on tex one may read the 
\href{http://www.ccas.ru/voron/download/voron05latex.pdf}{book} by K. Vorontsov.

\begin{enumerate}

\item Use decimal point as a separator: $3.14$ — good style, $3{,}14$ — bad style.
This goes against russian tradition, but favors copy-pasting numbers in software for computations. 
% \item Существует длинное тире, —, которое отличается от просто дефиса - и нужно,
% чтобы разделять части предложения, \href{https://ru.wikihow.com/напечатать-тире}{Инструкция
% в картинках по набору тире :)}
\item Use \verb|\[|\ldots\verb|\]| for display math formulas. Do not use \$\$\ldots\$\$!
\item Use \verb|cases| for systems of equations,
\verb|align*| for multiline formulas, \verb|enumerate| for enumerations.
\item Inside formulas use \verb|\text{|\ldots\} to write text.
\item Use \verb|\ldots| for ellipsis.
\item You can find useful macros in the preamble, like \verb|\P, \E, \Var, \Cov, \Corr, \cN|.
\item Use backslash before functions: \verb|\ln, \exp, \cos|\ldots
\item Use booktabs style for tables. You may use online \href{https://www.tablesgenerator.com}{tablesgenerator}. 
Choose booktabs table style instead of default table style.
\item Respect the letter ё! :)
\item Start every sentence in tex source from a new line. 
There will be no additional newlines in final pdf but tex file will be easier to read.
% \item В перечислениях после «Найдите» используй в качестве знаков препинания точки
% с запятой и точку в конце.
\item For multiplication use \verb|\cdot|. Please never use \verb|*| :)
\end{enumerate}

% стандарт имени файла:
% добавляется _sol в файле с решениями

% !TEX root = ../tssp_exams.tex

\newpage
\thispagestyle{empty}
\section{October exam}

\subsection[2022-2023]{\hyperref[sec:sol_kr_01_2022_2023]{2022-2023}}
\label{sec:kr_01_2022_2023} % \label{ссылка сюда}

Short rules: 120 minutes, online and offline. You may use one A4 cheat sheet.

Date: 2022-10-29

\begin{enumerate}

  \item {[10]} The random variables $X_i$ are independend and uniformly distributed on $[0;2]$.
  Find 
      \[
      \plim_{n\to\infty}  \frac{(X_1 - \bar X)^3 + (X_2 - \bar X)^3 + \ldots + (X_n - \bar X)^3}{n + 2022}.
      \]
      
  
  \item A Hedgehog starts at the point $x=2$ on the real line. 
  Every minute he moves one step left with probability $0.3$ or one step right with probability $0.7$.
  There are two exceptions from this rule: the absorbing point $x=0$ and the reflecting barrier at $x=3$.
  
  If the Hedgehog reaches the absorbing point $x=0$ then he stops moving and stays there. 
  If the Hedgehog reaches the reflecting barrier $x=3$ then his next move will be one step left with probability $1$.
  
  \begin{enumerate}
  \item {[2]} Write the transition matrix of this Markov chain. 
  \item {[3]} What is the probability that Hedgehog will be at $x=1$ after exactly 3 steps?
  \item {[5]} What is the expected time to reach the absorbing state?
  \end{enumerate}
  
  
  \item The random variables $X_i$ are independent and they take values $+1$ or $-1$ with equal probability. 
  
  \begin{enumerate}
  \item {[3]} Explicitely list all the events in sigma-algebra $\sigma(X_1 \cdot X_2)$.
  \item {[3]} Pavel says that he knows only whether $X_1$ and $X_3$ are equal. 
  How will you describe his knowledge with sigma-algebra?
  \item {[4]} How many events are in the sigma-algebra $\sigma(X_1, X_1 + X_2, X_1 + X_2 + X_3)$?
  \end{enumerate}
  
  
  \item Masha receives on average 10 sms per minute. Sms arrival is well described by the Poisson process. 
  
  \begin{enumerate}
  \item {[3]} What is the probability that Masha receives exactly 10 sms in the next 40 seconds?
  \item {[3]} Masha just received an sms. What is the probability that she will wait more that 2.5 seconds before the next one?
  \item {[4]} Find the covariance between the number of sms in the first 3 minutes and the number of sms in the first 10 minutes. 
  \end{enumerate}
  
  
  \item The random variables $X_i$ are independent and they take values $+1$ or $-1$ with equal probability. 
  
  \begin{enumerate}
  \item {[3]} Find $\E(X_3 \mid X_1, X_2)$, $\E(X_3 \mid X_1 + X_3)$.
  \item {[3]} Find $\Var(X_3 \mid X_1, X_2, X_3)$, $\Var(X_3 \mid X_1 + X_3)$.
  \item {[4]} Let $Y_n$ be equal to $\E(X_1 + \ldots +  X_{2022} \mid X_1, X_2, \ldots, X_n)$. \newline
  Is the process $Y_1$, $Y_2$, \ldots, $Y_{2022}$ a martingale?
  \end{enumerate}
  
  
  \item Consider a Wiener process $(W_t)$.
  \begin{enumerate}
      \item {[4]} Let $Y_t = t W_{2t}$. What is the distribution of $Y_t - Y_s$ for $t\geq s$? Is $Y_t$ a Wiener process?
      \item {[6]} Find a constant $\alpha$ such that $M_t = W_t^3 + \alpha t W_t$ is a martingale. 
  \end{enumerate}
  
  
  
  \end{enumerate}
  



\subsection[2021-2022]{\hyperref[sec:sol_kr_01_2021_2022]{2021-2022}}
\label{sec:kr_01_2021_2022} % \label{ссылка сюда}

Short rules: 120 minutes, online without proctoring. You may use any source you want but don't cheat.

Date: 2021-10-28

\begin{enumerate}

\item (10 points) Consider the Markov chain with the transition matrix
\[
  P = \begin{pmatrix}
    0.2 & 0.2 & 0 & 0.6 \\
    0.3 & 0.3 & 0.4 & 0 \\
    0 & 0 & 0.1 & 0.9 \\
    0 & 0 & 0.8 & 0.2 \\
  \end{pmatrix}.
\]

\begin{enumerate}
  \item (3 points) Split the chain in classes and classify them into closed or not closed.
  \item (2 points) Classify the states into recurrent or transient.
  \item (5 points) A Hedgehog starts in the state one and moves 
  randomly between states according to the transition matrix.

  What is the approximate probability that the Hedgehog will be in the 
  state four after $10^{2021}$ moves?
\end{enumerate}

Note: state number is the row (or column) number.

  \item (10 points) Gleb Zheglov catches one criminal every day. 
  With probability $0.2$ the catched criminal is replaced by $w$ new criminals. 
  Initially there are $n$ criminals in the town. 

  What is the expected time to the ultimate crime eradication in the town?

  \begin{enumerate}
    \item (4 points) Solve the problem for $w=1$ and $n=1$.
    \item (6 points) Solve the problem for arbitrary $w$ and $n$.
  \end{enumerate}

  \item (10 points) The random variables $X_i$ are independend and uniformly distributed on $[0;1]$.
  Find the probability limit
\[
\plim_{n\to\infty}  \max \left\{ \frac{\sum_{i=1}^n X_i}{n}, \frac{2\sum_{i=1}^n X^2_i}{n} \right\}.
\]


\item (10 points) Taxis arrive to the station according to the Poisson process with rate 1 per 5 minutes. 

Let $Y_t$ be the number of taxis that will arrive between 0 and $t$ minutes.

\begin{enumerate}
  \item (2 points) Sketch the expected value of $Y_t$ as a function of $t$.
  \item (8 points) Sketch the probability $\P(Y_t = Y_{60})$ as a function of $t$.
\end{enumerate}

Note: special points like intercepts or extrema should be explicitely marked.

\item (10 points) Prince Myshkin throws a fair coin until two consecutive heads appear. 
Let $N$ be the number of throws. 

Find the moment generating function of $N$. 

Hint: you may use the first step approach.

\item (20 points) Vincenzo Peruggia makes attempts to steal the Mona Lisa painting until the first 
success. 
Each attempt is successful with probability $0.1$.

Let $X$ be the number of attempts and $Z = \min\{X, 5\}$.

\begin{enumerate}
  \item (5 points) How many events are in sigma-algebras $\sigma(Z)$ and $\sigma(X)$?
  \item (5 points) If possible provide an example of events $A$ and $B$ such that: $A\in \sigma(Z)$ but $A\not\in\sigma(X)$; $B\in \sigma(X)$ but $B\not\in\sigma(Z)$.
  \item (10 points) Find $\E(Z \mid X)$ and $\E(X \mid Z)$.
\end{enumerate}






\end{enumerate}



% \subsection[что идет в оглавление]{\hyperref[на что ссылка]{текст ссылки}}
\subsection[2021-2022 retake]{\hyperref[sec:sol_kr_01_2021_2022_retake]{2021-2022 retake}}
\label{sec:kr_01_2021_2022_retake} % \label{ссылка сюда}


Short rules: 120 minutes, online without proctoring. You may use any source you want but don't cheat.

\begin{enumerate}

\item (10 points) Consider the Markov chain with the transition matrix
\[
  P = \begin{pmatrix}
    0.2 & 0.2 & 0 & 0.6 & 0 \\
    0.3 & 0.3 & 0.4 & 0 & 0\\
    0 & 0 & 0.3 & 0.7 & 0 \\
    0 & 0 & 0.8 & 0.2 & 0 \\
    0 & 0 & 0 & 0 & 1 \\
  \end{pmatrix}.
\]

\begin{enumerate}
  \item (3 points) Split the chain in classes and classify them into closed or not closed.
  \item (2 points) Classify the states into recurrent or transient.
  \item (5 points) A Hedgehog starts in the state one and moves 
  randomly between states according to the transition matrix.

  What is the approximate probability that the Hedgehog will be in the 
  state four after $10^{2021}$ moves?
\end{enumerate}

Note: state number is the row (or column) number.

  \item (10 points) Consider infinite ladder with steps numbered from $0$ to infinity. 
  I start at step $0$. Every day with probability $u$ I go one step up.
  With probability $d$ I go one step down. With probability $1-u-d$ I stay on the same step.

  If I am at step $0$ then I stay there with probability $1-u$ because it's impossible to go down. 

  Consider the case $d>u$. 
  
  What is the probability that I will be at step $0$ after $10^{1000}$ days?

  \item (10 points) The random variables $X_i$ are independend and uniformly distributed on $[0;2]$.
  Find the probability limit
\[
\plim_{n\to\infty}  \max \left\{ \frac{\sum_{i=1}^{10} X_i}{n}, \frac{\sum_{i=1}^n X^3_i}{n+1} \right\}.
\]


\item (10 points) Taxis arrive to the station according to the Poisson process with rate 1 per 5 minutes. 

Let $Y_t$ be the number of taxis that will arrive between 0 and $t$ minutes.

\begin{enumerate}
  \item (5 points) Sketch the probability $\P(Y_{t+3} = 1 \mid Y_t = 0)$ as a function of $t$.
  \item (5 points) Sketch the covariance $\Cov(Y_t, Y_{60})$ as a function of $t$.
\end{enumerate}

Note: special points like intercepts or extrema should be explicitely marked.

\item (10 points) The moment generating function of a random variable $X$ is $1/(1-2t)$.
\begin{enumerate}
    \item Find the moment generating function of $2X$.
    \item Find the moment generating function of $X + Y$ where $X$ and $Y$ are independent and identically distributed.
    \item Do you remember the sum of geometric progression? Find $\E(X^{2021})$.
\end{enumerate}

\item (20 points) Variables $X_1$, $X_2$, \ldots $X_{100}$ are independent and identically distributed
with mean $1$ and variance $2$. Each $X_i$ has only three possible values: 0, 1, and 2. 

\begin{enumerate}
  \item (5 points) How many events are in sigma-algebras $\sigma(X_1, X_2)$ and $\sigma(X_1 - X_2)$?
  \item (5 points) If possible provide an example of events $A$ and $B$ such that: $A\in \sigma(X_1, X_2)$ but $A\not\in\sigma(X_1 - X_2)$; $B\in \sigma(X_1 - X_2)$ but $B\not\in\sigma(X_1, X_2)$.
  \item (10 points) Find $\E(X_1 + \ldots + X_{100} \mid X_1 + \ldots + X_{50})$ and $\E(X_1 + \ldots + X_{50} \mid X_1 + \ldots + X_{100})$.
\end{enumerate}






\end{enumerate}





% \subsection[что идет в оглавление]{\hyperref[на что ссылка]{текст ссылки}}
\subsection[2020-2021]{\hyperref[sec:sol_kr_01_2020_2021]{2020-2021}}
\label{sec:kr_01_2020_2021} % \label{ссылка сюда}



Here $(W_t)$ denotes the standard Wiener process.

Date: 2020-10-30

\begin{enumerate}
    
    
    
    \item For $r<s<t<u$ find the following expected values 
    \begin{enumerate}
    \item $\E((W_u - W_t)^2(W_s - W_r)^2)$;
    \item $\E((W_u - W_s)(W_t - W_r))$;
    \item $\E((W_t - W_r)(W_s - W_r)^2)$;
    \item $\E(W_r W_s W_t)$;
    \item $\E(W_r W_s W_t \mid W_s)$;
    \end{enumerate}

\item Consider Ito process $X_t$

\[
dX_t = \exp(t) W_t\, dt + \exp(2W_t) \, dW_t, \quad X_0 = 1.
\]

Consider two processes, $A_t = 1 + t^2 + X_t^3$ and $B_t = 1 + t^2 + X_t^3 W_t^4$.

\begin{enumerate}
    \item Find $dA_t$ and $dB_t$.
    \item Write the corresponding explicit expressions for $A_t$ and $B_t$:
    \[
    const + \int_0^t \ldots dW_u + \int_0^t \ldots du
    \]
    \item Check whether $X_t$ is a martingale.
\end{enumerate}

\item Let $S_0 = 0$, $S_t = X_1 + X_2 + \ldots + X_t$. The increments $X_t$ are independent and identically distributed: 

\begin{tabular}{cccc}
\toprule
$x$ & $-1$ & $0$ & $1$ \\
$\P(X_t = x)$ & $0.2$ & $0.2$ & $0.6$ \\
\bottomrule
\end{tabular}

\begin{enumerate}
    \item If possible find all constants $a$ such that $M_t = S_t + at$ is a martingale.
  \item If possible find all constants $b$ such that $R_t = b^{S_t}$ is a martingale.
\end{enumerate}

\item Consider the process $X_t$

\[
X_t= tW_t + \int_0^t uW_u^2\, dW_u.
\]

\begin{enumerate}
    \item Find $\E(X_t)$, $\Var(X_t)$.
    \item Find $dX_t$.
    \item Check whether $X_t$ is a martingale.
\end{enumerate}

\item A Hedgehog in the fog starts in $(0, 0)$ at $t=0$ and moves randomly with equal probabilities in four directions (north, south, east, west) by one unit every minute. 

Let $X_t$ and $Y_t$ be his coordinates after $t$ minutes and $S_t = X_t + Y_t$.

\begin{enumerate}
    \item Find $\E(X_2 \mid S_2)$;
    \item Find $\Var(X_2 \mid S_2)$.
\end{enumerate}

Hint: $\Var(Y \mid X) = \E(Y^2 \mid X) - (\E(Y \mid X))^2$.

    \item Vampire Petr and Markov Chains. 
    
    Vampire Petr drinks blood of a new victim every day. 
    Unfortunately 20\% of the population are vaccinated against vampires. 
    If more than one victim of the last three victims are vaccinated then Petr will be instantaneously cured and will return to the normal life. 
 
    For simplicity let's assume that the last three victims were not vaccinated. 
    
    \begin{enumerate}
        \item What is the probability that vampire Petr will be cured in the next three days?
        \item How many victims will be bitten by vampire Petr on average?
    \end{enumerate}
 
    \item Vampire Boris and Martingales.
    
    To survive vampire Boris needs to bite 70 talented students. 
    
    These 70 talented students have formed a secret group. They have written their emails on small pieces of paper and have randomly distributed these pieces among them. Each student has exactly one piece of paper with an email\footnote{The group is so secret that it is possible that a student has his own email on his piece of paper}. 
    
    Initially vampire Boris knows contacts of just two persons from the group. Today he will contact them, drink their blood and get the emails they have. Then vampire Boris will contact new victims and so on.
    
    \begin{enumerate}
        \item For $t\geq 1$ consider the process $M_t$, the proportion of non bitten students after the day $t$. 
        
        Is this process a martingale?
        
        \item Using martingale stopping theorem or otherwise find the probability that vampire Boris will bite all 70 students. 
    \end{enumerate}
 
 
\end{enumerate}



% !TEX root = ../tssp_exams.tex
% all problems are copied to stochastic_pro 2024-11-03


% it's good to follow the strategy:
% after exams copy exam here and classify int stochastic_pro

\newpage
\thispagestyle{empty}
\section{December exam}
 
\subsection[2022-2023]{\hyperref[sec:sol_kr_02_2022_2023]{2022-2023}}
\label{sec:kr_02_2022_2023} % \label{ссылка сюда}



Short rules: 120 minutes, you may use two A4 cheat-sheets, offline + one online group.

\begin{enumerate}

\item Consider $X_t = \int_0^t W_u^3 dW_u + \int_0^t (W_u^3 + 3W_u u ) du - W_t^3 \cdot t$.

\begin{enumerate}
    \item Find $dX_t$ and the corresponding full form. 
    \item Is $X_t$ a martingale?
\end{enumerate}

\item Consider $X_t = \exp(-2W_t - 2t)$.
\begin{enumerate}
    \item Find $dX_t$. Is $X_t$ a martingale?
    \item Find $\E(X_t)$ and $\Var(X_t)$.
    \item Find $\int_0^t X_u dW_u$.
\end{enumerate}

\item As usual $(W_t)$ is a Wiener process.
\begin{enumerate}
    \item Find $\E(W_5 W_4 \mid W_4)$, $\Var(W_5 W_4 \mid W_4)$.
    \item Find covariance $\Cov(W_4 W_5, W_5 W_6)$.
\end{enumerate}

\item Let $X_i$ be independent identically distributed with $\P(X_i = 1) = 0.9$, $\P(X_i = -1 ) = 0.1$. 

Find all constants $a$ and $b$ such that $Y_t = a \exp\left(b\sum_{i=1}^t X_i\right)$ is a martingale. 

\item Consider two-period binomial model with initial share price $S_0 = 600$, 
Up and down multipliers are $u=1.2$, $d=0.9$, risk-free interest rate is $r = 0.05$ per period. 

Consider an option that pays you $X_2 = 100$ at $T=2$ if $S_2 > S_1$ and nothing otherwise. 

\begin{enumerate}
    \item Find the risk neutral probabilities. 
    \item Find the current price $X_0$ of the asset. 
    \item How much shares should I have at $t=1$ in the «up» state of the world to replicate the option?
\end{enumerate}

\item Consider Black and Scholes model with riskless rate $r$, volatility $\sigma$ and initial share price $S_0$. 

Find the current price $X_0$ of an option that pays you $X_2 = S_1^3$ at time $T=2$. 

\end{enumerate}




\subsection[2021-2022]{\hyperref[sec:sol_kr_02_2021_2022]{2021-2022}}
\label{sec:kr_02_2021_2022} % \label{ссылка сюда}

Short rules: 120 minutes, online without proctoring, $(W_t)$ is a standard Wiener process.

Date: 2021-12-25

\begin{enumerate}

  \item (10 points) Consider an Ito's process $I_t = 2022 + W_t t^2 + \int_0^t W_u^3 dW_u + \int_0^t W_u^2 du$.
    \begin{enumerate}
      \item Find $dI_t$ and check whether $I_t$ is a martingale. 
      \item Check whether $J_t = I_t - \E(I_t)$ is a martingale.
    \end{enumerate}
    
  \item (10 points) The random variables $(Z_t)$ are independent identically distributed 
  with moment generating function given by $M_{Z}(u) = 1/(1 - 5u)^3$. 
  
  We define $X_t$ as $X_t = \exp(Z_1 + 2Z_2 + 3Z_3 + \ldots + tZ_t)$ with $X_0 = 0$. 
  
  If possible find a martingale of the form $Y_t = h(t) X_t$ where $h()$ is a non-random function.
  
  \item (10 points) The process $(Z_t)$ in discrete time is called \textit{stationary} if it has constant expected value 
  and constant covariances $\gamma_k$ that do not depend on $t$. 
  \[
  \begin{cases}
  \E(Z_t) = \mu; \\
  \Cov(Z_t, Z_t) = \gamma_0; \\
  \Cov(Z_t, Z_{t+1}) = \gamma_1; \\
  \Cov(Z_t, Z_{t+2}) = \gamma_2; \\
  \ldots \\
  \end{cases}
  \]
  
  \begin{enumerate}
    \item If possible provide an example of a martingale that is not stationary.
    \item If possible provide an example of a stationary process that is not a martingale.
  \end{enumerate}
  
  \item (10 points) Find $\E(W_1 W_2 W_3)$ and $\E(W_2 W_3 \mid W_1)$.
  \item (10 points) Ded Moroz would like to receive $X_T = S^{-1}_T$ at time $T$ if $S_T < 1$ and nothing otherwise.
      
  Assume the framework of Black and Scholes model, $S_t$ is the share price, $r$ is the risk free rate,
  $\sigma$ is the volatility. 
  
  How much Ded Moroz should pay now at $t=0$?
  
  
  \item (20 points) Martingales are everywhere :)
  
  Consider the process $Y_t = \exp( - u W_t)$. 
  
  \begin{enumerate}
    \item Find a multiplier $h(u, t)$ such that $M_t = h(u, t) \cdot Y_t$ is a martingale. 
    \item Find $dY_t$, $\E(Y_t)$ and $\Var(Y_t)$.
    \item Consider $M_t$ that you have found as a function of $u$. 
    Find the Taylor approximation of the function $M_t(u)$ up to $u^4$. 
    \item Consider the coefficient before $u^4$ in the Taylor expansion of $M_t(u)$. 
    Is it a martingale?
  \end{enumerate}
  
  \item Bonus point. Guess your exam result (out of 70 possible points). 
  
  \end{enumerate}
  
      

% \subsection[что идет в оглавление]{\hyperref[на что ссылка]{текст ссылки}}
\subsection[2020-2021]{\hyperref[sec:sol_kr_02_2020_2021]{2020-2021}}
\label{sec:kr_02_2020_2021} % \label{ссылка сюда}



Today we celebrate Christmas Eve and 78 years of the Narkompros (People's Commissariat for Education) order governing the compulsory use of the letter «ё» in education process.

Date: 2020-12-24

\begin{enumerate}

    \item Ded Moroz would like to receive $S_1^3$ roubles at time $T=2$,
    where $S_t$ is the share price. Assume Black-Schёles model is valid, 
    the risk-free rate is $r=0.1$ and current share price is $S_0=100$.

    How much Ded Moroz should pay now at $t=0$?
    
    \item Consider stationary $AR(2)$ model, $y_t = 2 + 0.3 y_{t-1} - 0.02 y_{t-2} + u_t$, where $(u_t)$ is a white noise
    with $\Var(u_t) = 4$.
    
    The last two observations are $y_{100} = 2$, $y_{99} = 1$.
    \begin{enumerate}
        \item Find 95\% predictive interval for $y_{102}$.
        \item Find the first two values of the autocorrelation function, $\rho_1$, $\rho_2$.
        \item Find the first two values of the partial autocorrelation function, $\phi_{11}$, $\phi_{22}$.
    \end{enumerate}

    Hint: you need no more than 10 seconds to find both partial autocorrelations provided (b) is sёlved.

    \item The process $y_t$ is described by a simple $GARCH(1, 1)$ model:
    \[ 
        \begin{cases}
            y_t = \sigma_t \nu_t \\
            \sigma_{t}^{2}= 1 + 0.2 y_{t-1}^{2}+ 0.3 \sigma_{t-1}^{2}    \\
            \nu_t \sim \cN(0;1)
        \end{cases}     
    \]

    The variables $\nu_t$ are independent of past variables $y_{t-k}$, $\nu_{t-k}$, $\sigma_{t-k}$ for all $k\geq 1$.
    The prёcesses $y_t$, $\sigma^2_t$ are stationary. 


    Given $\sigma_{100}=1$ and $\nu_{100} = 0.5$ find 95\% predictive interval for $y_{102}$. 


    \item Snegurochka studies a stochastic analog of the Fibonacci sequence
    \[
        y_t = y_{t-1} + y_{t-2} + u_t,
    \]
    where $(u_t)$ is a white noise process. 
    \begin{enumerate}
        \item How many non-stationary solutions are there?
        \item What can you say about the number and the structure of the stationary solutions?
        \item Can Snёgurochka find two starting constants $y_0 = c_0$ and $y_1=c_1$ in such a way to make a solution stationary?
    \end{enumerate}

    Be brave! There are two more exercises!

    \item The semi-annual $y_t$ is modelled by $ETS(AAA)$ process:
    
    \[
    \begin{cases}
        u_t \sim \cN(0; 4) \\
        s_t = s_{t-2} + 0.1 u_t \\
        b_t = b_{t-1} + 0.2 u_t \\
        \ell_t = \ell_{t-1} + b_{t-1} + 0.3 u_t \\
        y_t = \ell_{t-1} + b_{t-1} + s_{t-2} + u_t \\
    \end{cases}    
    \]

    \begin{enumerate}
        \item Given that $s_{100} = 2$, $s_{99} = -1.9$, $b_{100} = 0.5$, $\ell_{100} = 4$ find 95\% prёdictive interval for $y_{102}$. 
        \item In this problem particular values of parameters are specified. And how many parameters are estimated in semi-annual $ETS(AAA)$ model before real forecasting?
    \end{enumerate}

    \item The variables $x_t$ take values $0$ or $1$ with equal probabilities.
    The variables $u_t$ are normal $\cN(0; 1)$. All variables are independent.
    
    Consider the process  $z_t = x_t (1-x_{t-2}) u_t$.

    \begin{enumerate}
        \item Find the covariance $\Cov(z_t, z_s)$. Is the process $z_t$ stationary?
        \item Given that $z_{100} = 2.3$ find shёrtest predictive intervals for $z_{101}$ and $z_{102}$ with probability of coverage at least 95\%.
    \end{enumerate}

    \item[Bёnus:] How many letters «ё» have you spotted?
    
 
\end{enumerate}

% !TEX root = ../tssp_exams.tex

\newpage
\thispagestyle{empty}
\section{April exam}


\subsection[2022-2023]{\hyperref[sec:sol_kr_03_2022_2023]{2022-2023}}
\label{sec:kr_03_2022_2023} % \label{ссылка сюда}

Short rules: 90 minutes, one A4 cheat sheet allowed. 

Date: 2023-03-25

\begin{enumerate}

    \item Consider $ETS(AAdN)$ model 
    \[
    \begin{cases}
    u_t  \sim \mathcal{N}(0;20) \\
    b_t = 0.9 b_{t-1} + 0.2 u_t \\
    \ell_t = \ell_{t-1} + 0.9 b_{t-1} + 0.3 u_t \\
    y_t = \ell_{t-1} + 0.9 b_{t-1} + u_t \\
    \end{cases}
    \]
    with $\ell_{100} = 20$ and $b_{100} = 1$.
    \begin{enumerate}
        \item Find 95\% prediction interval for $y_{102}$.
        \item Approximately find the best point forecast for $y_{10000}$.
    \end{enumerate}
    
    \item Consider the difference equation:
    \[
    y_t = 0.7y_{t-1} - 0.12 y_{t-2} + u_t,    
    \]
    where $(u_t)$ is a white noise. 
    \begin{enumerate}
        \item How many stationary and non-stationary solutions does the difference equation have?
    \end{enumerate}
    
    Consider stationary $AR(2)$ process that satisfies the difference equation. 
    
    \begin{enumerate}[resume]
        \item Find first two values of autocorrelation function.
        % \item Find first two values of partial autocorrelation function.
        \item Find $\alpha_1$ and $\alpha_2$ in $MA(\infty)$ representation 
    \[
    y_t = u_t + \alpha_1 u_{t-1} + \alpha_2 u_{t-2} + \alpha_3 u_{t-3} + \ldots
    \]
    \end{enumerate}
    
    
    \item The strictly stationary white noise $(u_t)$ follows $ARCH(1)$ model $\sigma^2_t = 3 + 0.5 u_{t-1}^2$ where 
    $u_t = \sigma_t \nu_t$ and $\nu_t \sim \mathcal{N}(0;1)$.
    \begin{enumerate}
        \item Find 95\% prediction interval for $u_{101}$ given that $u_{100} = -1$.
        \item Find $\E(u_t)$, $\Var(u_t)$.
        \item Find $\Corr(u_t, u_{t-1})$, $\Corr(u_t^2, u_{t-1}^2)$.
    \end{enumerate}
    
    \item The weight of a fish $Y_i$ is a discrete random variables with 
    distribution and observed frequencies given in the table 
    
    \begin{tabular}{cccc}
        \toprule
        Weight [kg] & 1 & 2 & 4 \\
        Probability & $0.2 + a$ & $0.3 - a$ & $0.5$ \\
        Observed frequency & $N_1$ & $N_2$ & $N_4$ \\
        \bottomrule
    \end{tabular}
    
    Fish weights $Y_i$ are independent. 
    
    \begin{enumerate}
        \item Find the maximum likelihood estimator of the parameter $a$. 
        \item Find the method of moments estimator of the parameter $a$. 
    \end{enumerate}
        
    \item You observe time between taxi arrivals on a stop, $Y_1$, $Y_2$, \ldots, $Y_n$.
    Assume that $Y_i$ are independent and exponentially distributed with $\E(Y_i) = \theta$, 
    that means the density of each $Y_i$ is $f(y) = \exp(-y/\theta)/\theta$ for $y\geq 0$. 
    Consider the following estimator of expected value
    \[
    \hat \theta = n \cdot \min\{Y_1, Y_2, \dots, Y_n \}    
    \]
    \begin{enumerate}
        \item Find the probability density function of $\hat \theta$. 
        \item Is $\hat \theta$ unbiased?
        \item Is $\hat \theta$ consistent?
    \end{enumerate}
    
    
\end{enumerate}
    



\subsection[2021-2022]{\hyperref[sec:sol_kr_03_2021_2022]{2021-2022}}
\label{sec:kr_03_2021_2022} % \label{ссылка сюда}

Short rules: 120 minutes, one A4 cheat sheet allowed. 

Date: 2022-04-04

% \textbf{Start exam by writing the following honor pledge and signing it.}
% \vspace{10pt}
% \begin{tcolorbox}
% I pledge on my honor that I will not give nor receive any 
% unauthorized assistance on this exam.
% \end{tcolorbox}
% \vspace{20pt}

% \textbf{Problems:}

\begin{enumerate}

\item Consider $ETS(AAN)$ model,
	$
	\begin{cases}
	y_t = \ell_{t-1} + b_{t-1} + u_t \\
	\ell_t = \ell_{t-1} + b_{t-1} + \alpha u_t \\
	b_t = b_{t-1} + \beta u_t \\
	u_t \sim \cN(0;\sigma^2). \\
	% s_t = s_{t-12} + \gamma \varepsilon_t \\
	\end{cases}
	$
		
Let $\ell_{100} = 50$, $b_{100} = 2$, $\alpha=0.4$, $\beta=0.5$, $\sigma^2 = 16$.

Calculate one step and two steps ahead 95\% predictive intervals. 

\item Consider the process $y_t = 4 + u_t + u_{t-1} + 2 u_{t-2}$, where $(u_t)$ is a white noise with variance $16$.

\begin{enumerate}
	\item Is this process stationary? Explain. 
	\item Find the autocorrelation function of this process. Explain the meaning of $\rho_2$.
	\item Consider the process $d_t = \Delta y_t$. Is it $ARIMA(p, d, q)$? If yes, then find $p$, $d$ and $q$.
\end{enumerate}

\item Consider the stationary $AR(2)$ process $y_t = 5 - 0.9y_{t-1} - 0.2y_{t-2} + u_t$, where $(u_t)$ is a white noise. 
\begin{enumerate}
	\item Find the first value of autocorrelation function $\rho_1$.
	\item Find the partial autocorrelation function of this process. Explain the meaning of $\phi_{22}$.
	\item What is the relationship between values of autocorrelation function $\rho_{100}$, $\rho_{99}$ and $\rho_{98}$.
\end{enumerate}

Hint: values $\phi_{22}$, $\phi_{33}$ etc may be calculated almost effortlessly :)

\item Consider iid sample from bivariate normal distribution, 
$
\begin{pmatrix}
	X_i \\
	Y_i \\
\end{pmatrix}	 \sim \cN \left(     
\begin{pmatrix}
	\theta \\
	2\theta \\
\end{pmatrix}; 
\begin{pmatrix}
	4 & 1 \\
	1 & 9 \\
\end{pmatrix}
\right).
$

Calculate Fischer information for the following cases: 
\begin{enumerate}
	\item You observe $X_1$ only. 
	\item You observe $X_1$, \ldots, $X_n$.
	\item You observe $X_1$, \ldots, $X_n$, $Y_1$, \ldots, $Y_n$.
\end{enumerate}

Hint: the multivariate normal density is 
$
f(u) = \frac{1}{\sqrt{\det(2\pi \Sigma)}} \exp( -\frac{1}{2}(u-\mu)^T \Sigma^{-1}(u-\mu)).
$

\item Random variables $X_1$, \ldots, $X_n$ are independent with density 
$
f(x) = \begin{cases}
	-\ln(a) \cdot a^x, \text{ if } x\geq 0, \\
	0, \text{ otherwise.}
\end{cases}	
$
\begin{enumerate}
	\item Estimate $a$ using maximum likelihood. 
	\item Check whether the estimator is unbiased and consistent. 
	\item Check whether the corresponding Cramer-Rao lower bound is attained. 
\end{enumerate}

\item Consider the $ARCH(1)$ model, $u_t = \sigma_t \nu_t$, where $\nu_t$ are iid $\cN(0;1)$ and 
$\sigma^2_t = 1 + 0.3 u_{t-1}^2$. 
\begin{enumerate}
	\item Find 95\% predictive interval for $u_{101}$ if $u_{100} = -2$.
	\item Find the autocorrelation function of $r_t = u_t^2$. 
\end{enumerate}

%\item Random variables $x_t$ are iid with $\P(x_t = 0) = \P(x_t = 1) = 0.5$. 
%Consider the process $r_t = x_t \cdot x_{t-1} - 0.25$.
%\begin{enumerate}
%	\item Is $(r_t)$ stationary?
%	\item Elon Musk promise that this is $MA(1)$ that can be rewritten as $r_t = u_t + \alpha u_{t-1}$.
	
%	Is Elon Musk right? If yes, then express $u_t$ using $x_t$ and its lagged values. 
%\end{enumerate}


\end{enumerate}

    

% \subsection[что идет в оглавление]{\hyperref[на что ссылка]{текст ссылки}}
\subsection[2020-2021]{\hyperref[sec:sol_kr_03_2020_2021]{2020-2021}}
\label{sec:kr_03_2020_2021} % \label{ссылка сюда}

Date: 2021-04-13, Rock 'N' Roll day


\textbf{Estimation questions}

\begin{enumerate}


    \item To go to the mountain top I use a gondola lift in the morning. 
    I go back from the top using the same gondola lift in the evening. 
    Cabins are numbered from $1$ to $a$. 

    I have noticed that the absolute difference of cabin numbers of my two trips was $10$. 

    \begin{enumerate}
        \item Estimate $a$ using maximum likelihood. 
        \item Estimate $a$ using method of moments. 
    \end{enumerate}

    \item Random variables $X_1$, $X_2$, \ldots,  $X_n$ are independent identically distributed with density 
    \[
    f(x_i \mid \lambda, a) = \frac{\lambda}{2} \exp(-\lambda \abs{x_i - a}).    
    \]

    Observed values for $n=3$ are $-3$, $1$, $11$.

    \begin{enumerate}
        \item Estimate $\lambda$ using method of moments for fixed $a = 1$. 
        \item Estimate $\lambda$ and $a$ using maximum likelihood.
    \end{enumerate}

    \item Random variables $X_1$, \ldots, $X_n$ are independent and normally distributed $\cN(1, 1/b)$. 
    
    \begin{enumerate}
        \item Estimate $b$ using maximum likelihood.
        \item Does the estimator achive the Cramer-Rao lower bound?
        \item Is the estimator consistent?
        \item Is the estimator unbiased?
    \end{enumerate}

    \item Random variables $X_1$, $X_2$, \ldots,  $X_n$ are independent identically distributed with density 
    \[
    f(x_i \mid \lambda) = \frac{\lambda}{2} \exp(-\lambda \abs{x_i}).    
    \]

    For $n=100$ I have 40 negative values with sum equal to $-300$ and 60 positive values with sum equal to $500$. 

    \begin{enumerate}
        \item Test the hypothesis $\lambda = 1$ using LR approach at significance level $\alpha=0.01$.
        \item Test the hypothesis $\lambda = 1$ using LM approach at significance level $\alpha=0.01$.
    \end{enumerate}


\end{enumerate}

    \textbf{Distribution questions}

    \begin{enumerate}[resume]
    \item I have three problems in the home assignment. 
    Time spent on each problem is modelled by independend exponentially distributed random variables with rate $\lambda$: $X_1$, $X_2$, $X_3$.

    \begin{enumerate}
        \item Find the moment generating function of $X_i$ and hence the moment generating function of $S = X_1 + X_2 + X_3$.
        \item Find $\E(S^3)$.
        \item Find the joint density of $R = X_1 / (X_1 + X_2 + X_3)$ and $S$.
    \end{enumerate}

    \item I have $100$ numbers written on small sheets of paper: $x_1$, $x_2$, \ldots, $x_{100}$. The sum of these numbers is $1$. 
    
    Find the possible values of the sum 
    \[
    \frac{x_1}{\sqrt{1-x_1}} +     \frac{x_2}{\sqrt{1-x_2}} + \ldots + \frac{x_{100}}{\sqrt{1-x_{100}}}.
    \]
    

    Hint: consider a randomly selected number $X$ and apply the Jensen's inequality.
    
 
\end{enumerate}

% !TEX root = ../tssp_exams.tex
% all problems are copied to stochastic_pro 2024-11-03


% it's good to follow the strategy:
% after exams copy exam here and classify int stochastic_pro

\newpage
\thispagestyle{empty}
\section{Final exam}

\subsection[2023-2024]{\hyperref[sec:sol_kr_04_2023_2024]{2023-2024}}
\label{sec:kr_04_2023_2024} % \label{ссылка сюда}

Short rules: 120 minutes, you may use one A4 cheat-sheet, offline

Date: 2024-04-27

\begin{enumerate}
    \item The variables $X_1$, \ldots, $X_n$ are independent identically distributed with density 
    \[
    f(x) = \begin{cases}
      \lambda \exp(-\lambda (x - \theta)), \text{ if } x\geq \theta \\
      0, \text{ otherwise}.
    \end{cases}  
    \]
    \begin{enumerate}
      \item {[5]} Find the method of moments estimator of $\lambda$ for known value $\theta = 1$ using the first moment. 
      \item {[5]} Find the method of moments estimator of $\lambda$ for unknown value $\theta$ using the first two moments. 
    \end{enumerate}

    \item {[10]} The variables $X_1$, \ldots, $X_n$ are independent and normally distributed $\cN(a, 2a)$.
  
    Find the maximum likelihood estimator of $a$.
  
    Hint: $f(x) = \frac{1}{\sqrt{2\pi \sigma^2}} \exp(-(x-\mu)^2/2\sigma^2)$.

    \item The variables $X_1$, \ldots, $X_n$ are independent and uniformly distributed $\dUnif[0;a]$ with $a>1$.
    We do not observe $X_i$ directly but we know whether each $X_i$ is larger than 1. 
    Hence we observe the indicators $Y_i = I(X_i > 1)$.
  
    Consider the estimator $\hat a = 1 / (1 - \bar Y)$.
  
    \begin{enumerate}
      \item {[5]} Is $\hat a$ consistent?
      \item {[5]} Is $\hat a$ unbiased for $n=2$?
    \end{enumerate}

    \item The variables $X_1$, \ldots, $X_n$ are independent and have Poisson distribution with intensity rate $\lambda$.
    In other words the probability mass function is given by $\P(X_i = k) = \exp(-\lambda) \lambda^k / k!$.
    
    \begin{enumerate}
      \item {[5]} Find theoretical Fisher information for $\lambda$ contained in the sample. 
      \item {[2]} Derive the maximum likelihood estimator for $\lambda$.
      \item {[3]} Does the maximum likelihood estimator attain the Cramer-Rao lower bound for variance? 
    \end{enumerate}

    \item {[10]} The variables $X_1$, \ldots, $X_n$ are independent and gamma distributed with density
    \[
    f(x) = \begin{cases}
      \lambda^\alpha x^{\alpha - 1} \exp(-\lambda x) / \Gamma(\alpha), \text{ if } x\geq 0 \\
      0, \text{ otherwise}.
    \end{cases}  
    \]
    \begin{enumerate}
      \item {[5]} Find a sufficient statistic for $\alpha$ if we know that $\lambda = 1$. 
      \item {[5]} Find a two dimensional sufficient statistic for unknown $\alpha$ and $\lambda$. 
    \end{enumerate}

    \item We have two independent random samples $X_1$, $X_2$, \ldots, $X_{n_x}$ and $Y_1$, $Y_2$, \ldots, $Y_{n_y}$.
    The random variables $X_i$ follow Poisson distribution with intensity rate $\lambda_x$, 
    random variables $Y_i$ follow Poisson distribution with intensity rate $\lambda_y$.
  
    We would like to test $H_0$: $\lambda_x = \lambda_y$ against $H_1$: $\lambda_x \neq \lambda_y$.
  
    \begin{enumerate}
      \item {[3]} Find the maximal value of log-likelihood under $H_0$.
      \item {[3]} Find the maximal value of log-likelihood under unrestricted model.
      \item {[2]} Construct the likelihood ratio test. 
      \item {[2]} Do you reject $H_0$ if $n_x = 100$, $n_y = 200$, $\sum x_i = 500$, $\sum y_i = 900$ at
      significance level $5\%$?
    \end{enumerate}
  
    Hint: chi-squared critical values for $\alpha = 0.05$ are $\chi^2_{df=1} = 3.84$, $\chi^2_{df=2} = 5.99$.
  

\end{enumerate}




\subsection[2022-2023]{\hyperref[sec:sol_kr_04_2022_2023]{2022-2023}}
\label{sec:kr_04_2022_2023} % \label{ссылка сюда}

Short rules: 120 minutes, you may use one A4 cheat-sheet, offline +  online.

Notes: $W_t$ denotes the standard Wiener process, 
you may use standard normal cumulative distribution function in your answers.

Date: Balalayka day, 2023-06-23.

\begin{enumerate}

\item The weight of a fish $Y_i$ is a discrete random variables with 
distribution and observed frequencies given in the table 

\begin{tabular}{cccc}
    \toprule
    Weight [kg] & 1 & 2 & $a$ \\
    Probability & $0.2 + 0.1a$ & $0.3 - 0.1a$ & $0.5$ \\
    Observed frequency & $N_1$ & $N_2$ & $N_a$ \\
    \bottomrule
\end{tabular}

Fish weights $Y_i$ are independent, $a > 10$ is unknown. 

\begin{enumerate}
    \item Find the method of moments estimator of the parameter $a$. 
    \item Find the maximum likelihood estimator of the parameter $a$. 
\end{enumerate}


\item The $ETS(AAdN)$ model is given by the system
    \[
    \begin{cases}
    u_t  \sim \mathcal{N}(0;20) \\
    b_t = 0.9 b_{t-1} + 0.2 u_t \\
    \ell_t = \ell_{t-1} + 0.9 b_{t-1} + 0.3 u_t \\
    y_t = \ell_{t-1} + 0.9 b_{t-1} + u_t \\
    \end{cases}
    \]
    with $\ell_{100} = 20$ and $b_{100} = 2$.
\begin{enumerate}
    \item Find conditional probability $\P(y_{102} > 30 \mid \ell_{100}, b_{100})$.
    \item Approximately find the best point forecast for $y_{10000}$.
\end{enumerate}
    
\item Stochastic process $X_t$ is defined by $X_t = 7 + u_t + 0.3 u_{t-1}$, where $(u_t)$ is a white noise 
with variance $\sigma^2$.
\begin{enumerate}
    \item Is $(X_t)$ stationary? 
    \item Find the autocorrelation function of $(X_t)$.
    \item Find $\E(X_{t+2} \mid X_t, X_{t-1}, \ldots)$.
\end{enumerate}

\item Consider the process $X_t = \int_0^t W_u^2 dW_u + \int_0^t (W_u^2 + 2W_u u ) du - W_t^2 \cdot t$.

\begin{enumerate}
    \item Find $dX_t$ and the corresponding full form. 
    \item Is $X_t$ a martingale?
    \item Find $\E(X_t)$.
\end{enumerate}


\item Consider the Black and Scholes model with riskless rate $r$, volatility $\sigma$ and initial share price $S_0$. 

Find the current price $X_0$ of an option that pays you one dollar at time $T=2$ only if $S_2 > \exp(3r) S_0$.

\item A hedgehog moves at random on the vertices $A$, $B$, $C$ and $D$ of a regular tetrahedron (тетраэдр).
She start at the vertex $A$ and every minute changes her position to one of the adjacent vertices with probability $1/3$
independently of past moves. 

\begin{enumerate}
    \item Write down the transition matrix of this Markov chain. 
    \item What is the expected time of the first return to the starting vertex $A$?
\end{enumerate}

\end{enumerate}






\subsection[2021-2022]{\hyperref[sec:sol_kr_04_2021_2022]{2021-2022}}
\label{sec:kr_04_2021_2022} % \label{ссылка сюда}

Short rules: 120 minutes, offline, one A4 cheat sheet allowed.

Date: 2022-06-25

\begin{enumerate}

\item Consider $ETS(ANN)$ model,
	$
	\begin{cases}
	y_t = \ell_{t-1} + u_t \\
	\ell_t = \ell_{t-1} + \alpha u_t \\
	u_t \sim \cN(0;\sigma^2). \\
	\end{cases}
	$
Let $\ell_{99} = 50$, $\alpha = 1/2$, $\sigma^2 = 16$, $y_{98} = 48$, $y_{99} = 52$, $y_{100} = 55$. Calculate 95\% predictive interval for $y_{101}$.

\item Young investor Winnie-the-Crypto compares two trading strategies: buying bitcoins from good bees and from bad bees. 
Let $d_t$ be the price difference at day $t$ (bad minus good). 
Winnie-the-Crypto would like to test $H_0$: $\E(d_t) = 0$ against $H_a$: $\E(d_t) \neq 0$ at $5\%$ significance level.

Winnie assumed that $(d_t)$ can be approximated by a $MA(1)$ process and estimated the parameters using $T=400$ observations, $\hat d_t = 2 + u_t + 0.7 u_{t-1}$ 
with $\hat\sigma^2_u = 4$.

\begin{enumerate}
	\item Estimate $\E(d_t)$, $\Var(d_t)$ and $\Cov(d_t, d_{t-1})$.
	\item Estimate $\E(\bar d)$, $\Var(\bar d)$ and help Winnie by considering $Z = \frac{\bar d - 0}{se(\bar d)}$.
\end{enumerate}



\item The variables $X_1$, \ldots, $X_n$ are independent and uniformly distributed on $[0; 2a]$ for some positive $a$. 

\begin{enumerate}
	\item Find any sufficient statistic for $a$. 
	\item How the answer will change if $X_i \sim U[-a; 2a]$?
\end{enumerate}


\item Consider an estimator $\hat a$ with $\E(\hat a) = 0.5a + 3$. For the given sample size the Fisher information is $I_F(a) = 400/a^2$.
\begin{enumerate}
	\item What is the theoretical minimal variance of $\hat a$?
	\item Assume that $\hat a$ attains the minimal variance boundary and is asymptotically normal. Given that $\hat a = 2022$ provide 95\% CI for $a$.
\end{enumerate}

\item You observe $X_1$, \ldots, $X_{400}$ and $Y_1$, \ldots, $Y_{400}$, $\bar X = 5$, $\bar Y = 6$. 
All variables are independent. 

Consider the null hypothesis that all random variables are exponentially distributed with common parameter $\lambda$ against alternative
that parameter is $\lambda_X$ for every $X_i$ and $\lambda_Y$ for every $Y_j$. 

\begin{enumerate}
	\item Estimate common $\lambda$ using maximum likelihood for the restricted model. 
	\item Estimate both $\lambda_X$ and $\lambda_Y$ using maximum likelihood in the unrestricted model. 
	\item Use LR-test to test the null hyphotesis at 5\% significance level. 
\end{enumerate}

\item The ultimate goal of this exercise is to prove the good upper bound for tail probability of a normal distribution: 
if $X\sim \cN(0; \sigma^2)$ then $\P(X > c) \leq \exp(-c^2/2\sigma^2)$.

Here are the guiding hints (you free to use not use them): 

\begin{enumerate}
	\item State the MGF of $X$. You may derive it or simply write it if you remember.
	\item Consider $Y = \exp(uX)$. Using Markov inequality provide the upper bound for $\P(Y > \exp(uc))$.
	\item Prove that $\P(X > c) \leq MGF_X(u)\exp(-uc)$ for any $u$.
	\item Find the value of $u$ that makes the upper bound as tight as possible. 
\end{enumerate}


\item (bonus) Draw good bees and bad bees selling crypto. Any funny statistics/math joke is also ok!

\end{enumerate}





% \subsection[что идет в оглавление]{\hyperref[на что ссылка]{текст ссылки}}
\subsection[2020-2021]{\hyperref[sec:sol_kr_04_2020_2021]{2020-2021}}
\label{sec:kr_04_2020_2021} % \label{ссылка сюда}

Today: $+31^{\circ}$,  World Refrigiration Day :)

You have 100 minutes. You can use A4 cheat sheet and calculator. Be brave! 

Date: 2021-06-26

\begin{enumerate}

\item I throw a fair die until the sequence 626 appears. Let $N$ be the number of throws.
\begin{enumerate}
    \item What is the expected value $\E(N)$?
    \item Write down the system of linear equations for the moment generating function of $N$. You don't need to solve it!
\end{enumerate}
    

\item Consider the following stationary process
\[
y_t = 1 + 0.5 y_{t-2} + u_t + u_{t-1},    
\]
where random variables $u_t$ are independent $\cN(0; 4)$.

\begin{enumerate}
    \item Find the 95\% predictive interval for $y_{101}$ given that $y_{100} = 2$, $y_{99} = 3$, $y_{98} = 1$, $u_{99} = -1$.
    \item Find the point forecast for $y_{101}$ given that $y_{100}=2$.
\end{enumerate}


\item I have an unfair coin with probability of heads equal to $h \in (0;1)$.
\begin{enumerate}
    \item Let $N$ be the number of tails before the first head. Find the MGF of $N$.
    \item Let $S$ be the number of tails before $k$ heads (not necessary consecutive). Find the MGF of $S$.
    \item What is the limit of $MGF_S(t)$ when $k \to \infty$ and $k \times h \to 0.5$? What is the name of the corresponding distribution?
\end{enumerate}


\item Consider the stochastic process $X_t = f(t) \cos (2021 W_t)$.
\begin{enumerate}
    \item Find $dX_t$.
    \item Find any $f(t) \neq 0$ such that $X_t$ is a martingale.
    \item Using $f(t)$ from the previous point find $\E(\cos (2021 W_t))$.
\end{enumerate}




\end{enumerate}


% \section{Ответы}
\addcontentsline{toc}{section}{Answer, hints and solutions} % руками добавляем фейковую секцию Ответы в оглавление
\addtocontents{toc}{\protect\setcounter{tocdepth}{0}}% Allow only \chapter in ToC


% !TEX root = ../tssp_exams.tex

\newpage
\thispagestyle{empty}
\section{October exam solutions}



\subsection[2023-2024]{\hyperref[sec:kr_01_2023_2024]{2023-2024}}
\label{sec:sol_kr_01_2023_2024} % \label{ссылка сюда}


\begin{enumerate}
    \item   
    \begin{enumerate}
      \item $\P(S_3 = B) = \P(A\to B\to A\to B) + \P(A\to C\to A\to B) + \P(A\to B\to C\to B)$
      \item 
      \[
      \begin{pmatrix}
        0 & 0.7 & 0.3 \\
        0.5 & 0 & 0.5 \\
        0.5 & 0.5 & 0 \\
      \end{pmatrix}  
      \]
      \item 
      \[
      \begin{cases}
        a = 0.5 b + 0.5 c \\
        b = 0.7 a + 0.5 c \\
        c = 0.3 a + 0.5 b \\
        a + b + c = 1  
      \end{cases}
      \]
      The solution is $a= 15/45$, $b=17/45$, $c=13/45$.
    \end{enumerate}
    


    \item
  
    \begin{enumerate}
      \item Consider $N$ as known fixed value, $\E(S \mid N) = N\E(X_1) = N\cdot 0.5$.
      First, let's find moment generating funciton for $X_i$:
      \[
      M_X(u) = \int_0^1 \exp(xu) \cdot 1 \, dx = \frac{\exp(u) - 1}{u}; 
      \]
      Hence $M_{S\mid N}(u) = (M_X(u))^N$ as $S$ is the sum of $N$ independent variables.
      \item Random variable $N$ is discrete, $M_S(u) = \P(N=1)(M_X(u))^1 + \P(N=2)(M_X(u))^2 + \ldots = \frac{0.7 M_X(u)}{ 1 -0.3M_X(u)}$.
      \item Moment generating function is used to calculate moments, $M_S''(0) - (M_S'(0))^2 = \Var(S)$.
    \end{enumerate}
  

    
    \item 
    Start with $X_0$: $\E(X_0) = 1$, $\Var(X_0) = 4/12 = 1/3$.

    \begin{enumerate}
      \item Expected value is constant, $\E(X_n) = 0.5 + 0.5 \E(X_{n-1})$, hence $\E(X_n) = 1$. 
      Variance goes to zero, $\Var(X_n) = 0.25 \Var(X_{n-1})$.
      \item $\plim X_n = 1$
    \end{enumerate}
    
    \item 
    Let's measure time in minutes. 
    \begin{enumerate}
      \item $\P(X_5 = 0) = \exp(-5 \lambda) = 0.05$, so $\lambda = \ln (0.05) / -5 = \ln(20) /5$.
      \item $\E(X_{180}) = 180 \lambda$, $\Var(X_{180}) = 180\lambda$
      \item $\P(X_{180} = 5) = \exp(-180 \lambda) (180\lambda)^5/5!$
    \end{enumerate}
  
    \item 
    \begin{enumerate}
      \item $Y = IX_1 + (1-I)X_2$
      \item Consider $I$ as known or fixed variable, $\E(Y \mid I) = I\E(X_1) + (1 - I)\E(X_2)$.
      Note that $I^2 = I$ and $(1-I)^2 = 1-I$, hence $\Var(Y \mid I) = I\Var(X_1) + (1-I)\Var(X_2)$.
      \item $\E(Y) = p \mu_1 + (1-p) \mu_2$ and $\Var(Y) = p(1-p)(\mu_1 - \mu_2)^2 + p\sigma^2_1 + (1-p)\sigma^2_2$,
      where $p = 0.3$, $\mu_1 = \sigma_1^2= 1$, $\mu_2 = \sigma_2^2 =2$.
    \end{enumerate}
  
    \item 
      \begin{enumerate}
       \item 
       Sigma-algebra: $\sigma(X) = \{\emptyset, \Omega, \{X=-2\},  \{X=0\},  \{X=2\},  \{X\neq -2\},  \{X\neq 0\},  \{X\neq 2\}\}$.
       Other descriptions are possible, for example, one may replace $\{X = -2\}$ by $\{X < 0\}$.
       \item Random variable $XY$ takes 3 distinct values, hence $\card \sigma(X \cdot Y) = 2^3 = 8$.
      \end{enumerate}
      
  \end{enumerate}
  


% \subsection[что идет в оглавление]{\hyperref[на что ссылка]{текст ссылки}}
\subsection[2022-2023]{\hyperref[sec:kr_01_2022_2023]{2022-2023}}
\label{sec:sol_kr_01_2022_2023} % \label{ссылка сюда}

\begin{enumerate}
    \item 
    \begin{align*}
        \plim \frac{\sum (X_i - \bar X)^3}{n+2022} = \plim \frac{\sum X_i^3 - 3\bar X\sum X_i^2 +3\bar X^2 \sum X_i - \bar X^3}{n+2022} = \\
        = \E(X_1^3) - 3\E(X_1^2) + 3\E(X_1) - 1 = 0;
    \end{align*}
    Note that $\E(X_1^2) = 4/3$, $\E(X_1^3) = 2$.
    \item $\P(X_3 = 1) = 0.3\cdot 0.7\cdot 0.3 + 0.7\cdot 1\cdot 0.3 = 0.21 \cdot 1.3$
    Let's denote $\tau_j = \min \{t \mid X_t = 0, X_0 = j\}$, $\mu_j = \E(\tau_j)$.
    \[
    \begin{cases}
        \mu_0 = 0 \\
        \mu_1 = 1 + 0.7\mu_2 \\
        \mu_2 = 1 + 0.3\mu_1 + 0.7\mu_3 \\
        \mu_3 = \mu_2 + 1 
    \end{cases}    
    \]
    We get $\mu_2 = 200/9$.
    \item \begin{enumerate}
        \item $\sigma(X_1 \cdot X_2) = \{\emptyset, \Omega, \{X_1 X_2 = 1\}, \{X_1 X_2 = -1\}\}$;
        \item Many answers are ok, for example $\sigma(X_1 X_3)$.
        \item Note that $\sigma(X_1, X_1 + X_2, X_1+ X_2+X_3) = \sigma(X_1, X_2, X_3)$, the number of events in sigma-algebra is
        $\card \sigma(X_1, X_1 + X_2, X_1+ X_2+X_3) = 2^8 = 256$.
    \end{enumerate}
    \item $\Cov(N_3, N_{10}) = \Cov(N_3, N_3 + (N_{10} - N_3)) = \Var(N_3) = 3\lambda$.
    \item $\E(X_3 \mid X_1, X_2) = \E(X_3) = 0$, $\E(X_3 \mid X_1 + X_3) = (X_1 + X_3) / 2$, $\Var(X_3 \mid X_1, X_3) = 0$,
    $\Var(X_3 \mid X_1 + X_3) = 1 - (X_1 + X_3)^2/4$.
    \item Посчитаем ожидание и получим $Y_n = X_1 + X_2 + \ldots + X_n$, the process $(Y_n)$ is a martingale.
    \item $\Var(Y_t - Y_s) = \Var(tW_{2t} - sW_{2s}) = 2t^3 + 2s^3 - 4ts^2$.
    We get $\E(M_{t+u} \mid \mathcal F_t ) = W_t^3 + 3W_t u  + \alpha(t + u) W_t$.
    From $\E(M_{t+u} \mid \mathcal F_t ) = W_t^3 + \alpha t W_t$ it follows that $\alpha = -3$.
\end{enumerate}


% \subsection[что идет в оглавление]{\hyperref[на что ссылка]{текст ссылки}}
\subsection[2021-2022]{\hyperref[sec:kr_01_2021_2022]{2021-2022}}
\label{sec:sol_kr_01_2021_2022} % \label{ссылка сюда}



\begin{enumerate}

\item 


\end{enumerate}
    

% \subsection[что идет в оглавление]{\hyperref[на что ссылка]{текст ссылки}}
\subsection[2021-2022 retake]{\hyperref[sec:kr_01_2021_2022_retake]{2021-2022 retake}}
\label{sec:sol_kr_01_2021_2022_retake} % \label{ссылка сюда}



\begin{enumerate}

\item 


\end{enumerate}



% \subsection[что идет в оглавление]{\hyperref[на что ссылка]{текст ссылки}}
\subsection[2020-2021]{\hyperref[sec:kr_01_2020_2021]{2020-2021}}
\label{sec:sol_kr_01_2020_2021} % \label{ссылка сюда}



\begin{enumerate}
    
    
    
    \item
 
 
\end{enumerate}



% !TEX root = ../tssp_exams.tex

\newpage
\thispagestyle{empty}
\section{December exam solutions}
\subsection[2022-2023]{\hyperref[sec:kr_02_2022_2023]{2022-2023}}
\label{sec:sol_kr_02_2022_2023} % \label{ссылка сюда} 

\begin{enumerate}
\item 
\begin{enumerate}
    \item $dX_t = (W^3_t-3W^2_t*t)dW_t$ (4 points), 1 point for comment how you get the answer (definition and Ito's lemma), 2 points for full form 
    \[ X_t = X_0 + \int_{0}^{t} W^3_u-3W^2_u*u \,dWu
    \]
    \item A process is a martingale as in short form $A_t dt =0$ (3 points)
\end{enumerate}

\item 
\begin{enumerate}
    \item $dX_t=-2X_t dW_t$ (2 points), this process is a martingale (1 point)
    \item $\E(X)=1$ (2 points), $\Var(X)=exp(4t)-1$ (2 points)
    \item \[ \int_{0}^{t} X_u\,dWu = \frac{1-X_t}{2}\] (3 points)
\end{enumerate}

\item \begin{enumerate}
    \item 2 points for $\E(W_5 W_4 | W_4) = W_4^2$, 3 points for $var(W_5 W_4 | W_4) = W_4^2$
    \item \begin{enumerate}
        \item 1-2 points for clever ideas
        \item 3 points for solution with serious mistakes
        \item 4 points for solutions with arithmetic errors
        \item 5 points for $cov(W_5 W_4, W_5 W_6) = 40$
    \end{enumerate}
\end{enumerate}

\item
 1-3 points depending on the cleverness of ideas.
 
 5 points if one got correct martingale:
 \[
 \E(Y_{t+1} | Y_t) = Y_t \E(e^{b X_{t+1}})
 \]    
10 points if one solved equation correctly:
\[
\E(e^{b X_{t+1}}) = 1 \rightarrow b = 0 or b = \ln(1/9)
\]
Minus 1 point if one forgot trivial solution a = 0 and b - any

\item \begin{enumerate}
    \item $p^*_u = p^*_d = 1/2$ (3 points)
    \item $X_1^u = X_1^d = (0.5\cdot 100 + 0.5\cdot 0) / 1.05$, hence $X_0 = 50/1.05^2 \approx 45.35$ (3 points)
    \item $\alpha = X_2^{uu} - X_2^{ud} / (S_2^{uu} - S_2^{ud}) = 100/216 \approx 0.46$ (4 points)
\end{enumerate}

\item 
You get 2 points almost for nothing:
\[
X_0 = \exp(-2r) \E_{*}(X_2)    
\]
Correct formula for $X_2$ in terms of $W_1^*$ gives your 4 points:
\[
X_2 = S_1^3 = S_0^3 \exp(3r)\exp(3\sigma W_1^* - 9\sigma^2/2).    
\]
Calculations of expected value (4 points more):
\[
X_0 = S_0^3 \exp(r)\exp(3\sigma^2).    
\]

\end{enumerate}

\subsection[2021-2022]{\hyperref[sec:kr_02_2021_2022]{2021-2022}}
\label{sec:sol_kr_02_2021_2022} % \label{ссылка сюда}



\begin{enumerate}

\item 


\end{enumerate}
    

% \subsection[что идет в оглавление]{\hyperref[на что ссылка]{текст ссылки}}
\subsection[2020-2021]{\hyperref[sec:kr_02_2020_2021]{2020-2021}}
\label{sec:sol_kr_02_2020_2021} % \label{ссылка сюда}



\begin{enumerate}
    
    
    
    \item
 
 
\end{enumerate}



% !TEX root = ../tssp_exams.tex

\newpage
\thispagestyle{empty}
\section{April exam solutions}
 

\subsection[2022-2023]{\hyperref[sec:kr_03_2022_2023]{2022-2023}}
\label{sec:sol_kr_03_2022_2023} % \label{ссылка сюда}

\begin{enumerate}
    \item 
\begin{enumerate}
    \item {[6 points]}
    \[
        y_{102}= \ell_{100} + (0.9 + 0.9^2) b_{100} + (0.3 + 0.18)u_{101} + u_{102}    
        \]
        \[
        (y_{102} \mid y_1, \ldots, y_{100}) ~ \cN(21.71, 24.608)    
        \]
        The interval
        \[
        [21.71 - 1.96 \cdot 4.96;21.71 + 1.96 \cdot 4.96]    
        \]
    \item {[4 points]}
    \[
    \lim_{h\to\infty} \E(y_{100+h} \mid y_1, \ldots, y_{100}) = \ell_{100} + (0.9 + 0.9^2 +\ldots) b_{100} = 20 + 9\cdot 1    
    \]    
\end{enumerate}
\item 
\begin{enumerate}
    \item {[2 points]} $\lambda_1 = 0.3$, $\lambda_2 = 0.4$, one stationary solution, infinitely many non-stationary solutions. 
    \item {[6 points]}: {[2 points] for the system} + {[2 points] for $\rho_1$} + {[2 points] for $\rho_2$}.
    \[
        \begin{cases}
            \gamma_1 = 0.7 \gamma_0  - 0.12 \gamma_1 \\
            \gamma_2 = 0.7 \gamma_1 - 0.12 \gamma_0. 
        \end{cases}
    \]
    \[
    \rho_1 = 70/112 = 0.625, \quad \rho_2 = 49/112 - 0.12 = 0.3175    
    \]
    \item {[2 points]}
    \[
    \alpha_1 = 0.7, \quad \alpha_2 = 0.37    
    \]
\end{enumerate}
\item 
\begin{enumerate}
    \item {[4 points]}
    \[
    \sigma^2_{101} = 3 + 0.5 (-1)^2= 3.5    
    \]
    \[
    (u_{101}\mid \sigma_{101}) \sim \cN(0; \sigma^2_{101})    
    \]
    \[
    [-1.96 \sqrt{3.5}; +1.96 \sqrt{3.5}]    
    \]
    \item {[3 points]} {[1 point]} for $\E(u_t)$ and {[2 points]} for $\Var(u_t)$ 
    The process $(u_t)$ is a white noise, hence
    \[
    \E(u_t) = 0.    
    \]
    \[
    \sigma^2_u = 3 + 0.5 \cdot \sigma^2_u    
    \]
    \item {[3 points]}: {[1 point]} for $\Corr(u_t, u_{t-1})$ and {[2 points]} for $\Corr(u_t^2, u_{t-1}^2)$ 
    The process $(u_t)$ is a white noise, hence
    \[
    \Corr(u_t, u_{t-1}) = 0.    
    \]
    \[
    u_t^2 = 3 + 0.5 u_{t-1}^2 + (u_t^2 - \sigma_t^2)    
    \]
    We notice that $r_t = u_t^2 - \sigma_t^2$ is a white noise, hence $u_t^2$ is an $AR(1)$ process.
    Hence, $\Corr(u_t^2, u_{t-1}^2) = 0.5$.
\end{enumerate}
\item 
\begin{enumerate}
    \item {[5 points]}
    \[
    L = const (0.2 + a)^{N_1} (0.3 - a)^{N_2} 0.5^{N_3}    
    \]
    \[
    \ell = const + N_1 \ln (0.2 +a) + N_2\ln(0.3 - a) + N_3 \ln 0.5    
    \]
    \[
    \frac{\partial \ell}{\partial a} = \frac{N_1}{0.2 + a} - \frac{N_2}{0.3 - a} 
    \]
    \[
    \hat{a}_{ML} = \frac{0.3 N_1 - 0.2 N_2}{N_1 + N_2}    
    \]
    We see that $\partial \ell /\partial a$ decreases as $a$ increases, so 
    $\hat{a}_{ML}$ is indeed the point of maximum. 
    \item {[5 points]}
    \[
    \E(Y_i) = (0.2 + a) + 2(0.3 - a) + 4\cdot 0.5 = 2.8 -a    
    \]
    \[
    \bar Y = \frac{N_1 + 2N_2  + 4N_4}{N_1 + N_2  + N_4}    
    \]
    \[
    \hat a_{MM} = 2.8 - \frac{N_1 + 2N_2  + 4N_4}{N_1 + N_2  + N_4}
    \]
\end{enumerate}
\item 
\begin{enumerate}
    \item {[6 points]}
    \[
    \P(\hat \theta > y) = \P(Y_1 > y/n)^n = \left( \exp(-y/n\theta) \right)^n = \exp(-y/\theta)
    \]
    Hence $\hat\theta$ has exponential distribution with rate $1/\theta$ and 
    probability density function
    \[
    f(t) = \begin{cases}
        \exp(-t/\theta)/\theta, \text{ if } t\geq 0, \\
        0, \text{ otherwise}.
    \end{cases}
    \]
    \item {[2 points]}
    
    The estimator is unbiased as
    \[
    \E(\hat\theta) = 1/(1/\theta) = \theta.    
    \]
    \item {[2 points]}
    
    The estimator is non consistent as its distribution does not depend on $n$.
\end{enumerate}

\end{enumerate}



\subsection[2021-2022]{\hyperref[sec:kr_03_2021_2022]{2021-2022}}
\label{sec:sol_kr_03_2021_2022} % \label{ссылка сюда}



\begin{enumerate}

\item 


\end{enumerate}
    

% \subsection[что идет в оглавление]{\hyperref[на что ссылка]{текст ссылки}}
\subsection[2020-2021]{\hyperref[sec:kr_03_2020_2021]{2020-2021}}
\label{sec:sol_kr_03_2020_2021} % \label{ссылка сюда}



\begin{enumerate}
    
    
    
    \item
 
 
\end{enumerate}



% !TEX root = ../tssp_exams.tex

\newpage
\thispagestyle{empty}
\section{Final exam solutions}


\subsection[2023-2024]{\hyperref[sec:kr_04_2023_2024]{2023-2024}}
\label{sec:sol_kr_04_2023_2024} % \label{ссылка сюда}



\begin{enumerate}

\item Let's observe that we may decompose $X_i$ as a sum $X_i = Y_i + \theta$,
where $Y_i \sim \dExp(\lambda)$. 

Hence, $\E(X_i) = 1/\lambda + \theta$, $\Var(X_i) = \Var(Y_i) = 1/\lambda^2$
and $\E(X_i^2) = 1/\lambda^2 + (1/\lambda + \theta)^2$.

There is an alternative solution with direct integration:
\[
\E(X_i) = \int_{\theta}^{+\infty} x f(x) \; dx, \quad \E(X_i^2) = \int_{\theta}^{+\infty} x^2 f(x) \; dx.
\]

\begin{enumerate}
    \item Solving $1/\hat\lambda + 1 = \bar X$ we obtain $\hat\lambda = 1/ (\bar X - 1)$.
    \item Solving for $\hat\lambda$ and $\hat\theta$ the system
\[
\begin{cases}
    1/\hat\lambda + \hat \theta = \bar X \\
    1/\hat\lambda^2 + (1/\hat\lambda + \hat\theta)^2 = M_2 \text{ with } M_2 = \sum X_i^2/n
\end{cases}    
\]
we obtain 
\[
\hat \lambda = \frac{1}{\sqrt{M_2 - \bar X^2}} , \quad \hat \theta =  \bar X - \sqrt{M_2 - \bar X^2}   
\]
\end{enumerate}

\item The log-likelihood function is equal to
\[
\ell(a) = \sum_{i=1}^n \left( (-0.5)\ln(4\pi) - 0.5 \ln a - (x_i - a)^2 / 4a \right).
\]
The equation $\ell'(a) = 0$ may be simplified to
\[
    n\hat a^2 + 2n\hat a - \sum X_i^2  = 0    
\]
Hence, 
\[
\hat a = \frac{-2n \pm \sqrt{4 n^2 + 4n \sum X_i^2}}{2n}
\]
We choose the root $\hat a > 0$ as $\Var(X_i) = 2a > 0$.
\[
\hat a = \sqrt{ 1  + \sum X_i^2/n} - 1
\]

Just for fun. 
In the case $X_i \sim \cN(a, ka)$ the equation would be
\[
    n\hat a^2 + k n\hat a - \sum X_i^2  = 0    
\]
And 
\[
\hat a = \frac{-nk + \sqrt{k^2 n^2  + 4n \sum X_i^2}}{2n}.    
\]



\item $\E(Y_i) = \P(X_i > 1) = (a-1) / a = p$.
\begin{enumerate}
    \item The estimator is consistent as
\[
\plim \hat a = \frac {1}{1- \plim \bar Y } = \frac{1}{1 - \frac{a-1}{a}} = a
\]
\item 
For $n=2$ we have the positive probability $p^2$ that $\bar Y = 1$.
Hence with positive probability $\hat a$ is not defined.
The value $\E(\hat a)$ does not exist for $n=2$.
\end{enumerate}

\item 
\begin{enumerate}
\item The log-likelihood function is equal to
\[
\ell(\lambda) = \sum_{i=1}^n \left( -\lambda + X_i \ln \lambda - \ln (X_i!)\right)    
\]
The score function is 
\[
\score(\lambda) = \ell'(\lambda) = \sum_{i=1}^n \left( -1 + X_i / \lambda\right).
\]
And
\[
\ell''(\lambda) = \sum_{i=1}^n \left(- X_i / \lambda^2 \right).
\]
Fisher information is 
\[
I_F = -\E(\ell''(\lambda)) = \sum \E(X_i) / \lambda^2 = n\lambda / \lambda^2 = n/\lambda.    
\]

\item Solving $\ell' = 0$ we obtain 
\[
\hat \lambda = \bar X    
\] 

\item Rewrite $\ell'(\lambda)$ using $\hat\lambda$. 
Be careful! Do not confound $\lambda$ and $\hat\lambda$. 
\[
\score(\lambda) = \ell'(\lambda) = -n + n \hat \lambda / \lambda.
\]
Hence the score function is linear function of $\hat\lambda$, $\Corr(\score(\lambda), \hat\lambda) = 1$
and the Cramer-Rao bound is attained. 

One may also  find $\E \hat \lambda = \lambda$, $\Var(\hat \lambda) = \lambda / n$ and
explicitly check that the general bound
\[
\Var(\hat \lambda) \geq 1/ I_F 
\]
is attained as equality in our case
\[
    \lambda / n = 1/(n/\lambda). 
\]


\end{enumerate}

\item We do not need the formula for $\Gamma(\alpha)$ here. 
\begin{enumerate}
    \item For known $\lambda = 1$ the likelihood is 
    \[
    L = \left(\prod X_i \right)^{\alpha - 1} \frac{1}{\Gamma(\alpha)} \cdot \exp( - \sum X_i).
    \]
    If we optimize this function for $\alpha$ the optimal $\hat\alpha$ will depend only on $\prod X_i$.
    Hence $\prod X_i$ is a sufficient statistic for $\alpha$. 
    There are many other sufficient statistics, $\sum \ln X_i$ is another example. 
    \item Now the likelihood is 
    \[
    L = \left(\prod X_i \right)^{\alpha - 1} \frac{1}{\Gamma(\alpha)} \lambda^{\alpha }\exp( - \lambda \sum X_i).
    \]
    If we optimize this function for $\alpha$ and $\lambda$ the optimal point will depend only on $\prod X_i$ and $\sum X_i$.
    Hence $\begin{pmatrix}
        \prod X_i & \sum X_i 
    \end{pmatrix}$ is a two dimensional sufficient statistic for $(\alpha, \lambda)$.
    
\end{enumerate}

\item 
\begin{enumerate}
    \item Under $H_0$ we have $X_i \sim \dPois(\lambda)$, $Y_i \sim \dPois(\lambda)$.
    \[
        \ell(\lambda) = \sum_{i=1}^{n_x} \left( -\lambda + X_i \ln \lambda - \ln (X_i!)\right) + 
        \sum_{i=1}^{n_y} \left( -\lambda + Y_i \ln \lambda - \ln (Y_i!)\right)       
    \]
    The score function is 
    \[
    \score(\lambda) = \ell'(\lambda) = \sum_{i=1}^{n_x} \left( -1 + X_i / \lambda\right) + \sum_{i=1}^{n_y} \left( -1 + Y_i / \lambda\right).
    \]
    The estimator is $\hat \lambda = (\sum X_i + \sum Y_i) / (n_x + n_y)$.
    \[
    \max \ell_R = - \hat \lambda (n_x + n_y) + \left(\sum X_i + \sum Y_i \right)\ln \hat\lambda - \sum \ln X_i! - \sum \ln Y_i!    
    \]
    
    \item In unrestricted model we have two independent estimators, 
    \[
    \hat \lambda_x = \bar X, \quad \hat \lambda_y = \bar Y    
    \]
    \[
    \max \ell_{UR} = - \hat \lambda_x n_x + \sum X_i \ln \hat\lambda_x + \sum Y_i \ln \hat\lambda_y - \sum \ln X_i! - \sum \ln Y_i!    
    \]

    \item 
    \[
    LR = 2(\max \ell_{UR} - \max \ell_R) = 2 \sum X_i (\ln \hat \lambda_x - \ln \hat\lambda) + 2 \sum Y_i (\ln \hat \lambda_y - \ln \hat\lambda)
    \]
    \item Unrestricted model has two parameters, restricted model has one parameter, 
    hence we use chi-squared disribution with $2 - 1 = 1$ degree of freedom, $LR_{\crit} = 3.84$.
    We calculate estimates, $\hat \lambda_x = 5$, $\hat\lambda_y = 4.5$, $\hat\lambda = 14/3$.
    
    \[
    LR = 1000 (\ln 5 - \ln (14/3)) + 1800 (\ln 4.5 - \ln (14/3)) \approx 3.5
    \]
    We do not reject $H_0$.
    


\end{enumerate}

\end{enumerate}



\subsection[2022-2023]{\hyperref[sec:kr_04_2022_2023]{2022-2023}}
\label{sec:sol_kr_04_2022_2023} % \label{ссылка сюда}



\begin{enumerate}

\item 


\end{enumerate}
    


\subsection[2021-2022]{\hyperref[sec:kr_04_2021_2022]{2021-2022}}
\label{sec:sol_kr_04_2021_2022} % \label{ссылка сюда}



\begin{enumerate}

\item 


\end{enumerate}
    

% \subsection[что идет в оглавление]{\hyperref[на что ссылка]{текст ссылки}}
\subsection[2020-2021]{\hyperref[sec:kr_04_2020_2021]{2020-2021}}
\label{sec:sol_kr_04_2020_2021} % \label{ссылка сюда}


\begin{enumerate}
\item Let's draw the chain

\begin{center}
\begin{tikzpicture}[->, >=stealth', auto, semithick, node distance=3cm]
\tikzstyle{every state}=[fill=white,draw=black,thick,text=black,scale=1]
\node[state]    (A)                     {$S$};
\node[state]    (B)[right of=A]   {$6$};
\node[state]    (C)[right of=B]   {$62$};
\node[state]    (D)[right of=C]   {$626$};
\path
(A) edge[loop below]     node{}         (A)
    edge                node{}     (B)
(B) edge                node{}           (C)
    edge[loop below]    node{}           (B)
    edge[bend right]    node{}           (A)
(C) edge                node{}           (D)
    edge[bend right]                node{}           (B)
    edge[bend right=40]         node{}           (A);
\end{tikzpicture}
\end{center}

The system of equations for expected values:
\[
\begin{cases}
x_s = 1 + \frac{1}{6} x_6 + \frac{5}{6} x_s \\
x_6 = 1 + \frac{1}{6} x_6 + \frac{1}{6} x_{62}  + \frac{4}{6} x_s \\
x_{62} = 1 + \frac{1}{6} \cdot 0 + \frac{1}{6} x_{6}  + \frac{4}{6} x_s \\
\end{cases}    
\]


The system of equations for moment generating functions:
\[
\begin{cases}
m_s(t) = \exp(t) \left(\frac{1}{6} m_6(t) + \frac{5}{6} m_s(t)\right) \\
m_6(t) = \exp(t) \left( \frac{1}{6} m_6(t) + \frac{1}{6} m_{62}(t)  + \frac{4}{6} m_s(t)\right) \\
m_{62}(t) = \exp(t) \left( \frac{1}{6} \cdot 1 + \frac{1}{6} m_{6}(t)  + \frac{4}{6} m_s(t) \right)\\
\end{cases}    
\]


\item \begin{enumerate}
    \item Let's denote by $x$ all available information, 
    \[
    x = \begin{pmatrix}
        y_{100} \\
        y_{99} \\
        y_{98} \\
        u_{99}
    \end{pmatrix}    
    \]
    Let's use $t=100$:
    \[
    y_{100} = 1 + 0.5 y_{98} + u_{100} + u_{99}    
    \]

    Using all available information we obtain $u_{100}  = 1.5$ and hence
    \[
    y_{101} \mid x \sim  \cN(1 + 0.5 y_{99} + u_{100} ; 4)
    \]

    \item Here we work with true betas:
    \[
    \E(y_{101} \mid y_{100}) = \mu_y + \frac{\Cov(y_{100}, y_{101})}{\Var(y_{100})}(y_{100} - \mu_y)    
    \]

\end{enumerate}
\item \begin{enumerate}
    \item Moment generating function
\[
m_N(t) = \sum_{j=0} \exp(tj) (1-h)^j h = h \sum_{j=0} (\exp(t) (1-h))^j = \frac{h}{1 - \exp(t) (1 - h)}  
\]
\item As $S = N_1 + N_2 + \ldots + N_k$:
\[
m_S(t) =  \left( \frac{h}{1 - \exp(t) (1 - h)} \right)^k
\]
\item Due to my mistake the limit is easy, $0$. 

In my dream it was $k\to \infty$, $k \cdot (1 - h) \to 0.5$ and that would be fun!

\end{enumerate}

\item \begin{enumerate}
    \item Let's use Ito's lemma
    \[
    dX_t = f'(t) \cos (2021 W_t) dt - 2021 f(t) \sin (2021 W_t) dW_t + \frac{1}{2}2021^2 f(t) \cos(2021 W_t) dt    
    \]
    \item To make $X_t$ a martingale we should kill $dt$ term. 
    \item As $X_t$ is martingale $\E(X_t) = \E(X_0) = f(0)$.
    So $\E(\cos (2021 W_t)) = f(0) / f(t)$.
\end{enumerate}

\end{enumerate}





\end{document}
