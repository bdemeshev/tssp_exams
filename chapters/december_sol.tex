% !TEX root = ../tssp_exams.tex

\newpage
\thispagestyle{empty}
\section{December exam solutions}
\subsection[2022-2023]{\hyperref[sec:kr_02_2022_2023]{2022-2023}}
\label{sec:sol_kr_02_2022_2023} % \label{ссылка сюда} 

\begin{enumerate}
\item 
\begin{enumerate}
    \item 5 points for $dX_t$, 2 points for full form
    \item 3 points: check that a process is a martingale
\end{enumerate}

\item 
\begin{enumerate}
    \item 2 points for $dX_t$, 1 point: check that a process is a martingale
    \item 2 points for $\E(X)$, 2 points for $\Var(X)$
    \item 3 points
\end{enumerate}

\item 
\item 

\item \begin{enumerate}
    \item $p^*_u = p^*_d = 1/2$ (3 points)
    \item $X_1^u = X_1^d = (0.5\cdot 100 + 0.5\cdot 0) / 1.05$, hence $X_0 = 50/1.05^2 \approx 45.35$ (3 points)
    \item $\alpha = X_2^{uu} - X_2^{ud} / (S_2^{uu} - S_2^{ud}) = 100/216 \approx 0.46$ (4 points)
\end{enumerate}

\item 
You get 2 points almost for nothing:
\[
X_0 = \exp(-2r) \E_{*}(X_2)    
\]
Correct formula for $X_2$ in terms of $W_1^*$ gives your 4 points:
\[
X_2 = S_1^3 = S_0^3 \exp(3r)\exp(3\sigma W_1^* - 9\sigma^2/2).    
\]
Calculations of expected value (4 points more):
\[
X_0 = S_0^3 \exp(r)\exp(3\sigma^2).    
\]

\end{enumerate}

\subsection[2021-2022]{\hyperref[sec:kr_02_2021_2022]{2021-2022}}
\label{sec:sol_kr_02_2021_2022} % \label{ссылка сюда}



\begin{enumerate}

\item 


\end{enumerate}
    

% \subsection[что идет в оглавление]{\hyperref[на что ссылка]{текст ссылки}}
\subsection[2020-2021]{\hyperref[sec:kr_02_2020_2021]{2020-2021}}
\label{sec:sol_kr_02_2020_2021} % \label{ссылка сюда}



\begin{enumerate}
    
    
    
    \item
 
 
\end{enumerate}


