% !TEX root = ../tssp_exams.tex

\newpage
\thispagestyle{empty}
\section{December exam solutions}
\subsection[2022-2023]{\hyperref[sec:kr_02_2022_2023]{2022-2023}}
\label{sec:sol_kr_02_2022_2023} % \label{ссылка сюда} 

\begin{enumerate}
\item 
\begin{enumerate}
    \item $dX_t = (W^3_t-3W^2_t\cdot t)dW_t$ (4 points), 1 point for comment how you get the answer (definition and Ito's lemma), 2 points for full form 
    \[ X_t = X_0 + \int_{0}^{t} W^3_u-3W^2_u\cdot u \,dWu
    \]
    \item A process is a martingale as in short form $A_t dt =0$ (3 points)
\end{enumerate}

\item 
\begin{enumerate}
    \item $dX_t=-2X_t dW_t$ (2 points), this process is a martingale (1 point)
    \item $\E(X)=1$ (2 points), $\Var(X)=\exp(4t)-1$ (2 points)
    \item \[ \int_{0}^{t} X_u\,dWu = \frac{1-X_t}{2}\] (3 points)
\end{enumerate}

\item \begin{enumerate}
    \item 2 points for $\E(W_5 W_4 | W_4) = W_4^2$, 3 points for $\Var(W_5 W_4 | W_4) = W_4^2$
    \item \begin{enumerate}
        \item 1-2 points for clever ideas
        \item 3 points for solution with serious mistakes
        \item 4 points for solutions with arithmetic errors
        \item 5 points for $\Cov(W_5 W_4, W_5 W_6) = 40$
    \end{enumerate}
\end{enumerate}

\item
 1-3 points depending on the cleverness of ideas.
 
 5 points if one got correct martingale:
 \[
 \E(Y_{t+1} | Y_t) = Y_t \E(e^{b X_{t+1}})
 \]    
10 points if one solved equation correctly:
\[
\E(e^{b X_{t+1}}) = 1 \rightarrow b = 0 or b = \ln(1/9)
\]
Minus 1 point if one forgot trivial solution a = 0 and b - any

\item \begin{enumerate}
    \item $p^*_u = p^*_d = 1/2$ (3 points)
    \item $X_1^u = X_1^d = (0.5\cdot 100 + 0.5\cdot 0) / 1.05$, hence $X_0 = 50/1.05^2 \approx 45.35$ (3 points)
    \item $\alpha = X_2^{uu} - X_2^{ud} / (S_2^{uu} - S_2^{ud}) = 100/216 \approx 0.46$ (4 points)
\end{enumerate}

\item 
You get 2 points almost for nothing:
\[
X_0 = \exp(-2r) \E_{*}(X_2)    
\]
Correct formula for $X_2$ in terms of $W_1^*$ gives your 4 points:
\[
X_2 = S_1^3 = S_0^3 \exp(3r)\exp(3\sigma W_1^* - 9\sigma^2/2).    
\]
Calculations of expected value (4 points more):
\[
X_0 = S_0^3 \exp(r)\exp(3\sigma^2).    
\]

\end{enumerate}

\subsection[2021-2022]{\hyperref[sec:kr_02_2021_2022]{2021-2022}}
\label{sec:sol_kr_02_2021_2022} % \label{ссылка сюда}



\begin{enumerate}

\item 


\end{enumerate}
    

% \subsection[что идет в оглавление]{\hyperref[на что ссылка]{текст ссылки}}
\subsection[2020-2021]{\hyperref[sec:kr_02_2020_2021]{2020-2021}}
\label{sec:sol_kr_02_2020_2021} % \label{ссылка сюда}



\begin{enumerate}
    
    
    
    \item
 
 
\end{enumerate}


