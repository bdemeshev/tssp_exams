% !TEX root = ../tssp_exams.tex

\newpage
\thispagestyle{empty}
\section{December exam solutions}


\subsection[2025-2026b]{\hyperref[sec:kr_02_2025_2026b]{2025-2026b}}
\label{sec:sol_kr_02_2025_2026b} % \label{ссылка сюда}

\begin{enumerate}
\item 
\item
\begin{enumerate}
    \item $\E(Y \mid Y - X) = (4 + Y - X)/2$ and $\Var(Y \mid Y - X) = (4 - \abs{Y - X})^2 / 12$.
    \item From $\cF$: $A_1 = \{ X > Y \}$, $A_2 = \{X > Y + 0.5\}$.
    Not from $\cF$: $B_1 = \{X > 2\}$, $B_2 = \{Y < 3\}$.
    \item Let's just add a known value $Y - X$ to the first term, $\Cov(X, Y \mid Y - X) = \Cov(X + (Y - X), Y \mid Y - X) = \Var(Y \mid Y - X)$.
\end{enumerate}
\item 
\begin{enumerate}
    \item $\alpha = -0.5$;
    \item $\beta = \ln 4 - \ln(e + e^{-1} + 2)$.
    \item $\E(T) = 2 \cdot 4^2 = 32$.
\end{enumerate}
\item
    \begin{enumerate}
      \item $\Corr(W_4 - W_2, W_3 - W_1) = 1 / \sqrt{2 \cdot 2} = 0.5$.
      \item Let's find best linear prediction of $W_3 - W_1$ given $W_4 - W_2$, $\beta = \Cov(W_4 - W_2, W_3 - W_1)/Var(W_4 - W_2) = 0.5$, hence $W_3 - W_1 = 0.5 (W_4 - W_2) + R$.
    
      It follows that 
      \[
      \E(W_3 - W_1 \mid W_4 - W_2) = 0.5 (W_4 - W_2)
      \]
      and that 
      \[
      \Var(W_3 - W_1 \mid W_4 - W_2) = \Var(R) = 2 - 0.25 \cdot 2 = 1.5. 
      \]
    \end{enumerate}
\item 
\begin{enumerate}
    \item $\E(A_t) = 0$, $\E(C_t) = t^3 / 3$.
    \item $\Var(B_t) = 15t^7$, $\Cov(A_t, C_t) = ...$.
\end{enumerate}
\item
From ordinary differential equation $g'(t) = -2026^2 g(t)$ we get for example, $g(t) = \exp(-2026^2/2 t)$.
\end{enumerate}

\subsection[2025-2026a]{\hyperref[sec:kr_02_2025_2026a]{2025-2026a}}
\label{sec:sol_kr_02_2025_2026a} % \label{ссылка сюда}

\begin{enumerate}
\item 
\item
\item 
\item
\begin{enumerate}
    \item $\E(A_t) = 0$, $\E(C_t) = t^2 /2$.
    \item $\Var(A_t) = t^3$, $\Cov(A_t, B_t) = 15t^4/4$, $\Var(C_t) = 7t^4/12 - (t^2/2)^2$.
    \[
    \E(C_t^2) = \int_0^t \int_0^t \E(W_u^2 W_s^2) \, du \, ds = 2 \int_{s=0}^t \int_{u=0}^s 2u^2 + us \, du \, ds = 2 \cdot 7/24 t^4.
    \]
\end{enumerate}
\item 
    \begin{enumerate}
      \item $\E(W_3^4) = 3^2 \cdot 3  = 27$, $W_3 = W_2 + I$ where $I \sim \cN(0, 1)$, hence $W_3^4 = (W_2 + I)^4$ and $\E(W_3^4 \mid W_2 = 10) = 10^4 + 6 \cdot 10^2 + 3$.
      \item Using Isserlis theorem, $\Cov(W_1 W_2, W_5 W_6) = \E(W_1 W_2 W_5 W_6) - \E(W_1 W_2) \E(W_5 W_6) = (5 + 2 + 2) - 5 = 4$.
    \end{enumerate}
\item
Let $Q_t = \int_0^t u^3 \, dW_u + \int_0^t 2u \, du$, so $R_t = \exp(Q_t)$.
By Ito's lemma:
\[
dR_t = \exp(Q_t) dQ_t + 0.5 \exp(Q_t) (dQ_t)^2 = \exp(Q_t)(t^3 \, dW_t + 2t \, dt) + 0.5 \exp(Q_t)t^6 \, dt.
\]
The differtial equation is $h'(t) = h(t) (2t + t^6/2)$.
\end{enumerate}

\subsection[2024-2025]{\hyperref[sec:kr_02_2024_2025]{2024-2025}}
\label{sec:sol_kr_02_2024_2025} % \label{ссылка сюда} 




\subsection[2023-2024]{\hyperref[sec:kr_02_2023_2024]{2023-2024}}
\label{sec:sol_kr_02_2023_2024} % \label{ссылка сюда} 

\begin{enumerate}
    \item 
 \begin{enumerate}
    \item {[3]} $S_t$ is not a martingale, $\E(S_{t+1} \mid \cF_t) = S_t + 0.6 - 0.4 \neq S_t$.
    \item {[7]} Two solutions, one is trivial $c=1$, the other\dots 
 \end{enumerate}

\item {[10 points]}

\begin{enumerate}
    \item {[4]} $dM_t = -3a(t)\sin(3W_t)dW_t + a'(t)\cos(3W_t)dt - 0.5 \cdot 9 a(t) \cos(3W_t) dt$ 
    \item {[6]} If $(M_t)$ is a martingale then $a'(t) - 4.5 a(t) = 0$, hence $a(t) = c\cdot \exp(4.5t)$.
\end{enumerate}
 
\item {[10 points]} 
\begin{enumerate}
    \item {[3]} $\E(S_n) = \rho t$.
    \item {[4]} $\Var(S_n) \to 0$
    \item {[2]} $S_n \to \rho t$ in mean squared sense. 
    \item {[1]} $dA_t dB_t = \rho dt$
\end{enumerate}

\item 
\begin{enumerate}
    \item {[2]} $(X_t)$ is not a martingale, as we have $W_t dt$ term;
    \item {[4]} 
    \[
    d(X_t W_t) = W_t dX_t + X_t dW_t + dX_t dW_t  = (W_t^3 dW_t + W_t^2 dt) + X_t dW_t + W_t^2 dt = (W_t^3 + X_t) dW_t + 2W_t^2 dt
    \]
    \item {[4]} $\Cov(X_t, W_t) = \E(X_t W_t) - 0$,
    \[
    X_t W_t = 0 + \int_0^t (W_u^3 + X_u) dW_u + \int_0^t 2W_u^2 du
    \]
    \[
    \E(X_t W_t) =  \int_0^t \E(2W_u^2) du =  \int_0^t 2u du = t^2.
    \]
\end{enumerate}

\item {[10 points]} 

\item {[10 points]}  
\[
X_0 = \exp(-rT)\E^*(X_T) = \exp(-rT) \P^*(S_T \geq 2S_0) = \dots
\]

\end{enumerate}


\subsection[2022-2023]{\hyperref[sec:kr_02_2022_2023]{2022-2023}}
\label{sec:sol_kr_02_2022_2023} % \label{ссылка сюда} 

\begin{enumerate}
\item 
\begin{enumerate}
    \item $dX_t = (W^3_t-3W^2_t\cdot t)dW_t$ (4 points), 1 point for comment how you get the answer (definition and Ito's lemma), 2 points for full form 
    \[ 
    X_t = X_0 + \int_{0}^{t} W^3_u-3W^2_u\cdot u \,dWu
    \]
    \item A process is a martingale as in short form $A_t dt =0$ (3 points)
\end{enumerate}

\item 
\begin{enumerate}
    \item $dX_t=-2X_t dW_t$ (2 points), this process is a martingale (1 point)
    \item $\E(X)=1$ (2 points), $\Var(X)=\exp(4t)-1$ (2 points)
    \item \[ \int_{0}^{t} X_u\,dWu = \frac{1-X_t}{2}\] (3 points)
\end{enumerate}

\item \begin{enumerate}
    \item 2 points for $\E(W_5 W_4 \mid W_4) = W_4^2$, 3 points for $\Var(W_5 W_4 \mid W_4) = W_4^2$
    \item \begin{enumerate}
        \item 1-2 points for clever ideas
        \item 3 points for solution with serious mistakes
        \item 4 points for solutions with arithmetic errors
        \item 5 points for $\Cov(W_5 W_4, W_5 W_6) = 40$
    \end{enumerate}
\end{enumerate}

\item
 1-3 points depending on the cleverness of ideas.
 
 5 points if one got correct martingale:
 \[
 \E(Y_{t+1} \mid Y_t) = Y_t \E(e^{b X_{t+1}})
 \]    
10 points if one solved equation correctly:
\[
\E(e^{b X_{t+1}}) = 1 \rightarrow b = 0 or b = \ln(1/9)
\]
Minus 1 point if one forgot trivial solution $a = 0$ and any $b$.

\item \begin{enumerate}
    \item $p^*_u = p^*_d = 1/2$ (3 points)
    \item $X_1^u = X_1^d = (0.5\cdot 100 + 0.5\cdot 0) / 1.05$, hence $X_0 = 50/1.05^2 \approx 45.35$ (3 points)
    \item $\alpha = X_2^{uu} - X_2^{ud} / (S_2^{uu} - S_2^{ud}) = 100/216 \approx 0.46$ (4 points)
\end{enumerate}

\item 
You get 2 points almost for nothing:
\[
X_0 = \exp(-2r) \E_{*}(X_2)    
\]
Correct formula for $X_2$ in terms of $W_1^*$ gives your 4 points:
\[
X_2 = S_1^3 = S_0^3 \exp(3r)\exp(3\sigma W_1^* - 9\sigma^2/2).    
\]
Calculations of expected value (4 points more):
\[
X_0 = S_0^3 \exp(r)\exp(3\sigma^2).    
\]

\end{enumerate}

\subsection[2021-2022]{\hyperref[sec:kr_02_2021_2022]{2021-2022}}
\label{sec:sol_kr_02_2021_2022} % \label{ссылка сюда}



\begin{enumerate}

\item 


\end{enumerate}
    

% \subsection[что идет в оглавление]{\hyperref[на что ссылка]{текст ссылки}}
\subsection[2020-2021]{\hyperref[sec:kr_02_2020_2021]{2020-2021}}
\label{sec:sol_kr_02_2020_2021} % \label{ссылка сюда}



\begin{enumerate}
    
    
    
    \item
 
 
\end{enumerate}


