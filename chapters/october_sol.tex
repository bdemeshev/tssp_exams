% !TEX root = ../tssp_exams.tex

\newpage
\thispagestyle{empty}
\section{October exam solutions}



\subsection[2023-2024]{\hyperref[sec:kr_01_2023_2024]{2023-2024}}
\label{sec:sol_kr_01_2023_2024} % \label{ссылка сюда}


\begin{enumerate}
    \item  $\P(A \to B) = 0.7$, 
    $\P(A \to C) = 0.3$, $\P(B \to C) = \P(B \to A) = 0.5$,  $\P(C \to B) = \P(C \to A) = 0.5$.
  
    \begin{enumerate}
      \item $\P(S_3 = B) = \P(A\to B\to A\to B) + \P(A\to C\to A\to B) + \P(A\to B\to C\to B)$
      \item 
      \[
      \begin{pmatrix}
        0 & 0.7 & 0.3 \\
        0.5 & 0 & 0.5 \\
        0.5 & 0.5 & 0 \\
      \end{pmatrix}  
      \]
      \item 
      \[
      \begin{cases}
        a = 0.5 b + 0.5 c \\
        b = 0.7 a + 0.5 c \\
        c = 0.3 a + 0.5 b \\
        a + b + c = 1  
      \end{cases}
      \]
      The solution is $a= 15/45$, $b=17/45$, $c=13/45$.
    \end{enumerate}
    


    \item
  
    \begin{enumerate}
      \item Consider $N$ as known fixed value, $\E(S \mid N) = N\E(X_1) = N\cdot 0.5$.
      First, let's find moment generating funciton for $X_i$:
      \[
      M_X(u) = \int_0^1 \exp(xu) \cdot 1 \, dx = \frac{\exp(u) - 1}{u}; 
      \]
      Hence $M_{S\mid N}(u) = (M_X(u))^N$ as $S$ is the sum of $N$ independent variables.
      \item $M_S(u) = \P(N=1)(M_X(u))^1 + \P(N=2)(M_X(u))^2 + \ldots = \frac{0.7 M_X(u)}{ 1 -0.3M_X(u)}$.
      \item Moment generating function is used to calculate moments, $M_S''(0) - (M_S'(0))^2 = \Var(S)$
    \end{enumerate}
  

    
    \item 
    Start with $X_0$: $\E(X_0) = 1$, $\Var(X_0) = 4/12 = 1/3$.

    \begin{enumerate}
      \item Expected value is constant, $\E(X_n) = 0.5 + 0.5 \E(X_{n-1})$, hence $\E(X_n) = 1$. 
      Variance goes to zero, $\Var(X_n) = 0.25 \Var(X_{n-1})$.
      \item $\plim X_n = 1$
    \end{enumerate}
    
    \item 
    Let's measure time in minutes. 
    \begin{enumerate}
      \item $\P(X_5 = 0) = \exp(-5 \lambda) = 0.05$, so $\lambda = \ln (0.05) / -5 = \ln(20) /5$.
      \item $\E(X_{180}) = 180 \lambda$, $\Var(X_{180}) = 180\lambda$
      \item $\P(X_{180} = 5) = \exp(-180 \lambda) (180\lambda)^5/5!$
    \end{enumerate}
  
    \item 
    \begin{enumerate}
      \item $Y = IX_1 + (1-I)X_2$
      \item Consider $I$ as known or fixed variable, $\E(Y \mid I) = I\E(X_1) + (1 - I)\E(X_2)$.
      Note that $I^2 = I$ and $(1-I)^2 = 1-I$, hence $\Var(Y \mid I) = I\Var(X_1) + (1-I)\Var(X_2)$.
      \item $\E(Y) = p \mu_1 + (1-p) \mu_2$ and $\Var(Y) = p(1-p)(\mu_1 - \mu_2)^2 + p\sigma^2_1 + (1-p)\sigma^2_2$,
      where $p = 0.3$, $\mu_1 = \sigma_1^2= 1$, $\mu_2 = \sigma_2^2 =2$.
    \end{enumerate}
  
    \item 
      \begin{enumerate}
       \item 
       Sigma-algebra: $\sigma(X) = \{\emptyset, \Omega, \{X=-2\},  \{X=0\},  \{X=2\},  \{X\neq -2\},  \{X\neq 0\},  \{X\neq 2\}\}$.
       Other descriptions are possible, for example, one may replace $\{X = -2\}$ by $\{X < 0\}$.
       \item Random variable $XY$ takes 3 distinct values, hence $\card \sigma(X \cdot Y) = 2^3 = 8$.
      \end{enumerate}
      
  \end{enumerate}
  


% \subsection[что идет в оглавление]{\hyperref[на что ссылка]{текст ссылки}}
\subsection[2022-2023]{\hyperref[sec:kr_01_2022_2023]{2022-2023}}
\label{sec:sol_kr_01_2022_2023} % \label{ссылка сюда}

\begin{enumerate}
    \item 
    \begin{align*}
        \plim \frac{\sum (X_i - \bar X)^3}{n+2022} = \plim \frac{\sum X_i^3 - 3\bar X\sum X_i^2 +3\bar X^2 \sum X_i - \bar X^3}{n+2022} = \\
        = \E(X_1^3) - 3\E(X_1^2) + 3\E(X_1) - 1 = 0;
    \end{align*}
    Note that $\E(X_1^2) = 4/3$, $\E(X_1^3) = 2$.
    \item $\P(X_3 = 1) = 0.3\cdot 0.7\cdot 0.3 + 0.7\cdot 1\cdot 0.3 = 0.21 \cdot 1.3$
    Let's denote $\tau_j = \min \{t \mid X_t = 0, X_0 = j\}$, $\mu_j = \E(\tau_j)$.
    \[
    \begin{cases}
        \mu_0 = 0 \\
        \mu_1 = 1 + 0.7\mu_2 \\
        \mu_2 = 1 + 0.3\mu_1 + 0.7\mu_3 \\
        \mu_3 = \mu_2 + 1 
    \end{cases}    
    \]
    We get $\mu_2 = 200/9$.
    \item \begin{enumerate}
        \item $\sigma(X_1 \cdot X_2) = \{\emptyset, \Omega, \{X_1 X_2 = 1\}, \{X_1 X_2 = -1\}\}$;
        \item Many answers are ok, for example $\sigma(X_1 X_3)$.
        \item Note that $\sigma(X_1, X_1 + X_2, X_1+ X_2+X_3) = \sigma(X_1, X_2, X_3)$, the number of events in sigma-algebra is
        $\card \sigma(X_1, X_1 + X_2, X_1+ X_2+X_3) = 2^8 = 256$.
    \end{enumerate}
    \item $\Cov(N_3, N_{10}) = \Cov(N_3, N_3 + (N_{10} - N_3)) = \Var(N_3) = 3\lambda$.
    \item $\E(X_3 \mid X_1, X_2) = \E(X_3) = 0$, $\E(X_3 \mid X_1 + X_3) = (X_1 + X_3) / 2$, $\Var(X_3 \mid X_1, X_3) = 0$,
    $\Var(X_3 \mid X_1 + X_3) = 1 - (X_1 + X_3)^2/4$.
    \item Посчитаем ожидание и получим $Y_n = X_1 + X_2 + \ldots + X_n$, the process $(Y_n)$ is a martingale.
    \item $\Var(Y_t - Y_s) = \Var(tW_{2t} - sW_{2s}) = 2t^3 + 2s^3 - 4ts^2$.
    We get $\E(M_{t+u} \mid \mathcal F_t ) = W_t^3 + 3W_t u  + \alpha(t + u) W_t$.
    From $\E(M_{t+u} \mid \mathcal F_t ) = W_t^3 + \alpha t W_t$ it follows that $\alpha = -3$.
\end{enumerate}


% \subsection[что идет в оглавление]{\hyperref[на что ссылка]{текст ссылки}}
\subsection[2021-2022]{\hyperref[sec:kr_01_2021_2022]{2021-2022}}
\label{sec:sol_kr_01_2021_2022} % \label{ссылка сюда}



\begin{enumerate}

\item 


\end{enumerate}
    

% \subsection[что идет в оглавление]{\hyperref[на что ссылка]{текст ссылки}}
\subsection[2021-2022 retake]{\hyperref[sec:kr_01_2021_2022_retake]{2021-2022 retake}}
\label{sec:sol_kr_01_2021_2022_retake} % \label{ссылка сюда}



\begin{enumerate}

\item 


\end{enumerate}



% \subsection[что идет в оглавление]{\hyperref[на что ссылка]{текст ссылки}}
\subsection[2020-2021]{\hyperref[sec:kr_01_2020_2021]{2020-2021}}
\label{sec:sol_kr_01_2020_2021} % \label{ссылка сюда}



\begin{enumerate}
    
    
    
    \item
 
 
\end{enumerate}


