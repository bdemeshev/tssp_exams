% !TEX root = ../tssp_exams.tex

\newpage
\thispagestyle{empty}
\section{April exam}



\subsection[2023-2024]{\hyperref[sec:sol_kr_03_2023_2024]{2023-2024}}
\label{sec:kr_03_2023_2024} % \label{ссылка сюда}

Short rules: 120 minutes, one A4 cheat sheet allowed. 

Date: 2024-03-05

\begin{enumerate}

\item Consider $MA(2)$ process given by 
\[
y_t = 5 + u_t + 2u_{t-1} + 4 u_{t-2},
\]
where $(u_t)$ is a white noise with $\Var(u_t) = \sigma^2$.

\begin{enumerate}
\item {[1]} Find the expected value $\E(y_t)$.
\item {[7]} Find the autocorrelation function $\rho_k = \Corr(y_t, y_{t-k})$.
\item {[2]} Is the process $(y_t)$ stationary?
\end{enumerate}

\item Consider $MA(2)$ process given by 
\[
y_t = 5 + u_t + 2u_{t-1} + 4 u_{t-2},
\]
where $u_t$ are normal independent random variables with $\Var(u_t) = 4$.

You know that $u_{100} = 2$ and $u_{99} = -1$.

\begin{enumerate}
\item {[5]} Find the 95\% predictive interval for $y_{101}$.
\item {[5]} Find the 95\% predictive interval for $y_{1000001}$.
\end{enumerate}

\item The stationary process $(y_t)$ has autocorrelation function $\rho_k = 0.2^k$ and expected value $100$.
\begin{enumerate}
\item {[7]} Find the first two values of the partial autocorrelation function, $\phi_{11}$ and $\phi_{22}$.
\item {[3]} Provide a possible linear recurrence equation for this process. 
Your equation may include $y_t$, its lags and a white noise process $(u_t)$.
\end{enumerate}

\item Consider the equation $y_t = 5 + 2.5 y_{t-1} - y_{t-2} + u_t$, where $(u_t)$ is a white noise process. 
\begin{enumerate}
\item {[3]} Find the roots of the corresponding characteristic equation. 
\item {[4]} Rewrite the process as $A(L)(y_t - \mu) = u_t$. 
You should explicitely write the lag polynomial $A(L)$ and the value of $\mu$.
\item {[1]} How many non-stationary solutions does the equation have?
\item {[1]} How many stationary solutions does the equation have?
\item {[1]} How many stationary solutions of the $MA(\infty)$ form with respect to $(u_t)$ does the equation have?
\end{enumerate}

\item {[10]} The semi-annual $(y_t)$ is modelled by $ETS(ANA)$ process:

\[
\begin{cases}
    u_t \sim \cN(0; 4) \\
    s_t = s_{t-2} + 0.1 u_t \\
    \ell_t = \ell_{t-1} + 0.3 u_t \\
    y_t = \ell_{t-1} + s_{t-2} + u_t \\
\end{cases}    
\]

Given that $s_{100} = 3$, $s_{99} = -2$, $\ell_{100} = 100$ find 95\% predictive interval for $y_{102}$. 


\item {[10]} The semi-annual $(y_t)$ is modelled by $ETS(ANA)$ process:

\[
\begin{cases}
    u_t \sim \cN(0; 4) \\
    s_t = s_{t-2} + 0.1 u_t \\
    \ell_t = \ell_{t-1} + 0.3 u_t \\
    y_t = \ell_{t-1} + s_{t-2} + u_t \\
    \ell_0 = 100, s_0 = -3, s_{-1} = 3 \\
\end{cases}    
\]

Check whether the process $(y_t)$ is stationary. 

    

\end{enumerate}



\subsection[2022-2023]{\hyperref[sec:sol_kr_03_2022_2023]{2022-2023}}
\label{sec:kr_03_2022_2023} % \label{ссылка сюда}

Short rules: 90 minutes, one A4 cheat sheet allowed. 

Date: 2023-03-25

\begin{enumerate}

    \item Consider $ETS(AAdN)$ model 
    \[
    \begin{cases}
    u_t  \sim \cN(0;20) \\
    b_t = 0.9 b_{t-1} + 0.2 u_t \\
    \ell_t = \ell_{t-1} + 0.9 b_{t-1} + 0.3 u_t \\
    y_t = \ell_{t-1} + 0.9 b_{t-1} + u_t \\
    \end{cases}
    \]
    with $\ell_{100} = 20$ and $b_{100} = 1$.
    \begin{enumerate}
        \item Find 95\% prediction interval for $y_{102}$.
        \item Approximately find the best point forecast for $y_{10000}$.
    \end{enumerate}
    
    \item Consider the difference equation:
    \[
    y_t = 0.7y_{t-1} - 0.12 y_{t-2} + u_t,    
    \]
    where $(u_t)$ is a white noise. 
    \begin{enumerate}
        \item How many stationary and non-stationary solutions does the difference equation have?
    \end{enumerate}
    
    Consider stationary $AR(2)$ process that satisfies the difference equation. 
    
    \begin{enumerate}[resume]
        \item Find first two values of autocorrelation function.
        % \item Find first two values of partial autocorrelation function.
        \item Find $\alpha_1$ and $\alpha_2$ in $MA(\infty)$ representation 
    \[
    y_t = u_t + \alpha_1 u_{t-1} + \alpha_2 u_{t-2} + \alpha_3 u_{t-3} + \ldots
    \]
    \end{enumerate}
    
    
    \item The strictly stationary white noise $(u_t)$ follows $ARCH(1)$ model $\sigma^2_t = 3 + 0.5 u_{t-1}^2$ where 
    $u_t = \sigma_t \nu_t$ and $\nu_t \sim \mathcal{N}(0;1)$.
    \begin{enumerate}
        \item Find 95\% prediction interval for $u_{101}$ given that $u_{100} = -1$.
        \item Find $\E(u_t)$, $\Var(u_t)$.
        \item Find $\Corr(u_t, u_{t-1})$, $\Corr(u_t^2, u_{t-1}^2)$.
    \end{enumerate}
    
    \item The weight of a fish $Y_i$ is a discrete random variables with 
    distribution and observed frequencies given in the table 
    
    \begin{tabular}{cccc}
        \toprule
        Weight [kg] & 1 & 2 & 4 \\
        Probability & $0.2 + a$ & $0.3 - a$ & $0.5$ \\
        Observed frequency & $N_1$ & $N_2$ & $N_4$ \\
        \bottomrule
    \end{tabular}
    
    Fish weights $Y_i$ are independent. 
    
    \begin{enumerate}
        \item Find the maximum likelihood estimator of the parameter $a$. 
        \item Find the method of moments estimator of the parameter $a$. 
    \end{enumerate}
        
    \item You observe time between taxi arrivals on a stop, $Y_1$, $Y_2$, \ldots, $Y_n$.
    Assume that $Y_i$ are independent and exponentially distributed with $\E(Y_i) = \theta$, 
    that means the density of each $Y_i$ is $f(y) = \exp(-y/\theta)/\theta$ for $y\geq 0$. 
    Consider the following estimator of expected value
    \[
    \hat \theta = n \cdot \min\{Y_1, Y_2, \dots, Y_n \}    
    \]
    \begin{enumerate}
        \item Find the probability density function of $\hat \theta$. 
        \item Is $\hat \theta$ unbiased?
        \item Is $\hat \theta$ consistent?
    \end{enumerate}
    
    
\end{enumerate}
    



\subsection[2021-2022]{\hyperref[sec:sol_kr_03_2021_2022]{2021-2022}}
\label{sec:kr_03_2021_2022} % \label{ссылка сюда}

Short rules: 120 minutes, one A4 cheat sheet allowed. 

Date: 2022-04-04

% \textbf{Start exam by writing the following honor pledge and signing it.}
% \vspace{10pt}
% \begin{tcolorbox}
% I pledge on my honor that I will not give nor receive any 
% unauthorized assistance on this exam.
% \end{tcolorbox}
% \vspace{20pt}

% \textbf{Problems:}

\begin{enumerate}

\item Consider $ETS(AAN)$ model,
	$
	\begin{cases}
	y_t = \ell_{t-1} + b_{t-1} + u_t \\
	\ell_t = \ell_{t-1} + b_{t-1} + \alpha u_t \\
	b_t = b_{t-1} + \beta u_t \\
	u_t \sim \cN(0;\sigma^2). \\
	% s_t = s_{t-12} + \gamma \varepsilon_t \\
	\end{cases}
	$
		
Let $\ell_{100} = 50$, $b_{100} = 2$, $\alpha=0.4$, $\beta=0.5$, $\sigma^2 = 16$.

Calculate one step and two steps ahead 95\% predictive intervals. 

\item Consider the process $y_t = 4 + u_t + u_{t-1} + 2 u_{t-2}$, where $(u_t)$ is a white noise with variance $16$.

\begin{enumerate}
	\item Is this process stationary? Explain. 
	\item Find the autocorrelation function of this process. Explain the meaning of $\rho_2$.
	\item Consider the process $d_t = \Delta y_t$. Is it $ARIMA(p, d, q)$? If yes, then find $p$, $d$ and $q$.
\end{enumerate}

\item Consider the stationary $AR(2)$ process $y_t = 5 - 0.9y_{t-1} - 0.2y_{t-2} + u_t$, where $(u_t)$ is a white noise. 
\begin{enumerate}
	\item Find the first value of autocorrelation function $\rho_1$.
	\item Find the partial autocorrelation function of this process. Explain the meaning of $\phi_{22}$.
	\item What is the relationship between values of autocorrelation function $\rho_{100}$, $\rho_{99}$ and $\rho_{98}$.
\end{enumerate}

Hint: values $\phi_{22}$, $\phi_{33}$ etc may be calculated almost effortlessly :)

\item Consider iid sample from bivariate normal distribution, 
$
\begin{pmatrix}
	X_i \\
	Y_i \\
\end{pmatrix}	 \sim \cN \left(     
\begin{pmatrix}
	\theta \\
	2\theta \\
\end{pmatrix}; 
\begin{pmatrix}
	4 & 1 \\
	1 & 9 \\
\end{pmatrix}
\right).
$

Calculate Fischer information for the following cases: 
\begin{enumerate}
	\item You observe $X_1$ only. 
	\item You observe $X_1$, \ldots, $X_n$.
	\item You observe $X_1$, \ldots, $X_n$, $Y_1$, \ldots, $Y_n$.
\end{enumerate}

Hint: the multivariate normal density is 
$
f(u) = \frac{1}{\sqrt{\det(2\pi \Sigma)}} \exp( -\frac{1}{2}(u-\mu)^T \Sigma^{-1}(u-\mu)).
$

\item Random variables $X_1$, \ldots, $X_n$ are independent with density 
$
f(x) = \begin{cases}
	-\ln(a) \cdot a^x, \text{ if } x\geq 0, \\
	0, \text{ otherwise.}
\end{cases}	
$
\begin{enumerate}
	\item Estimate $a$ using maximum likelihood. 
	\item Check whether the estimator is unbiased and consistent. 
	\item Check whether the corresponding Cramer-Rao lower bound is attained. 
\end{enumerate}

\item Consider the $ARCH(1)$ model, $u_t = \sigma_t \nu_t$, where $\nu_t$ are iid $\cN(0;1)$ and 
$\sigma^2_t = 1 + 0.3 u_{t-1}^2$. 
\begin{enumerate}
	\item Find 95\% predictive interval for $u_{101}$ if $u_{100} = -2$.
	\item Find the autocorrelation function of $r_t = u_t^2$. 
\end{enumerate}

%\item Random variables $x_t$ are iid with $\P(x_t = 0) = \P(x_t = 1) = 0.5$. 
%Consider the process $r_t = x_t \cdot x_{t-1} - 0.25$.
%\begin{enumerate}
%	\item Is $(r_t)$ stationary?
%	\item Elon Musk promise that this is $MA(1)$ that can be rewritten as $r_t = u_t + \alpha u_{t-1}$.
	
%	Is Elon Musk right? If yes, then express $u_t$ using $x_t$ and its lagged values. 
%\end{enumerate}


\end{enumerate}

    

% \subsection[что идет в оглавление]{\hyperref[на что ссылка]{текст ссылки}}
\subsection[2020-2021]{\hyperref[sec:sol_kr_03_2020_2021]{2020-2021}}
\label{sec:kr_03_2020_2021} % \label{ссылка сюда}

Date: 2021-04-13, Rock 'N' Roll day


\textbf{Estimation questions}

\begin{enumerate}


    \item To go to the mountain top I use a gondola lift in the morning. 
    I go back from the top using the same gondola lift in the evening. 
    Cabins are numbered from $1$ to $a$. 

    I have noticed that the absolute difference of cabin numbers of my two trips was $10$. 

    \begin{enumerate}
        \item Estimate $a$ using maximum likelihood. 
        \item Estimate $a$ using method of moments. 
    \end{enumerate}

    \item Random variables $X_1$, $X_2$, \ldots,  $X_n$ are independent identically distributed with density 
    \[
    f(x_i \mid \lambda, a) = \frac{\lambda}{2} \exp(-\lambda \abs{x_i - a}).    
    \]

    Observed values for $n=3$ are $-3$, $1$, $11$.

    \begin{enumerate}
        \item Estimate $\lambda$ using method of moments for fixed $a = 1$. 
        \item Estimate $\lambda$ and $a$ using maximum likelihood.
    \end{enumerate}

    \item Random variables $X_1$, \ldots, $X_n$ are independent and normally distributed $\cN(1, 1/b)$. 
    
    \begin{enumerate}
        \item Estimate $b$ using maximum likelihood.
        \item Does the estimator achive the Cramer-Rao lower bound?
        \item Is the estimator consistent?
        \item Is the estimator unbiased?
    \end{enumerate}

    \item Random variables $X_1$, $X_2$, \ldots,  $X_n$ are independent identically distributed with density 
    \[
    f(x_i \mid \lambda) = \frac{\lambda}{2} \exp(-\lambda \abs{x_i}).    
    \]

    For $n=100$ I have 40 negative values with sum equal to $-300$ and 60 positive values with sum equal to $500$. 

    \begin{enumerate}
        \item Test the hypothesis $\lambda = 1$ using LR approach at significance level $\alpha=0.01$.
        \item Test the hypothesis $\lambda = 1$ using LM approach at significance level $\alpha=0.01$.
    \end{enumerate}


\end{enumerate}

    \textbf{Distribution questions}

    \begin{enumerate}[resume]
    \item I have three problems in the home assignment. 
    Time spent on each problem is modelled by independend exponentially distributed random variables with rate $\lambda$: $X_1$, $X_2$, $X_3$.

    \begin{enumerate}
        \item Find the moment generating function of $X_i$ and hence the moment generating function of $S = X_1 + X_2 + X_3$.
        \item Find $\E(S^3)$.
        \item Find the joint density of $R = X_1 / (X_1 + X_2 + X_3)$ and $S$.
    \end{enumerate}

    \item I have $100$ numbers written on small sheets of paper: $x_1$, $x_2$, \ldots, $x_{100}$. The sum of these numbers is $1$. 
    
    Find the possible values of the sum 
    \[
    \frac{x_1}{\sqrt{1-x_1}} +     \frac{x_2}{\sqrt{1-x_2}} + \ldots + \frac{x_{100}}{\sqrt{1-x_{100}}}.
    \]
    

    Hint: consider a randomly selected number $X$ and apply the Jensen's inequality.
    
 
\end{enumerate}
